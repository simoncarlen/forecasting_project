%\documentclass[12pt, twoside]{report}
\documentclass[12pt]{report}

\usepackage[utf8]{inputenc}
\usepackage{graphicx}
\graphicspath{ {figures/} }
\usepackage[font=small,labelfont=bf]{caption}
%\usepackage{caption}
 \usepackage{lscape}

%\usepackage[a4paper,width=150mm,top=25mm,bottom=25mm,bindingoffset=6mm]{geometry}
\usepackage[a4paper, width=155mm, top=24mm, bottom=24mm]{geometry}

\usepackage{amsmath}
\usepackage{mathtools}
\numberwithin{equation}{section}

\usepackage{booktabs} % table package
%\usepackage{tabularx} % table package for width
\usepackage{lscape}
\usepackage{fancyhdr}
\usepackage[nottoc,numbib]{tocbibind} % references in table of contents
\usepackage[euler]{textgreek}
\usepackage{upgreek}
\usepackage{varioref}
\usepackage{hyperref}
\usepackage[capitalise]{cleveref}
\usepackage[hang,flushmargin]{footmisc} % no indent in footnote

\usepackage{siunitx}
%\usepackage{amsmath}
\usepackage{float} % use H as argument to force figure placement
\usepackage[dvipsnames]{xcolor} % color in text
\usepackage{bm}

\usepackage{chngcntr}
\usepackage{titlesec} % allowing subsubsections




%\pagestyle{fancy}
%\fancyhead{}
%\fancyhead[LE, RO]{Research project}
%\fancyfoot{}
%\fancyfoot[LE, RO]{\thepage}
%\renewcommand{\headrulewidth}{0.4pt}
%\renewcommand{\footrulewidth}{0.4pt}
%\showthe\textwidth

\usepackage{cite}

% keeping report format without including chapters
% https://tex.stackexchange.com/questions/62516/how-to-suppress-chapter-in-chapter-while-keeping-numbering
\makeatletter
\def\@makechapterhead#1{%
  \vspace*{50\p@}%
  {\parindent \z@ \raggedright \normalfont
    \interlinepenalty\@M
    \Huge\bfseries  \thechapter.\quad #1\par\nobreak
    \vskip 40\p@
  }}
\makeatother


\begin{document}

% TITLEPAGE

\begin{titlepage}
    \begin{center}
        \vspace*{2cm}
            
        \Huge
%        \textbf{Forecasting Air Pollution With Machine Learning}
        \textbf{A Statistical and Machine Learning Approach to Air Pollution Forecasts}
            
        \vspace{0.25cm}
        
        \LARGE
            
        \vspace{1cm}
            
        \textbf{Simon Carlén}
            
        %\vfill
        \vspace{250pt plus 1pt minus 1pt}
            
            
        %\includegraphics[width=0.3\textwidth]{university}
         
        	\vspace{.75cm}
	\large

	\begin{minipage}{0.65\textwidth}
	Degree project, 15 credits \\\indent
	Computer and Systems Sciences \\\indent
	Degree project at the master level \\\indent
	Spring term 2022 \\\indent 
	Supervisor: Sindri Magnússon \\\indent
	Co-supervisor: Ali Beikmohammadi \\\indent
	Reviewer: Petter Karlström
%	Swedish title: Luftföroreningsprognoser med djupinlärning
	\end{minipage}
	\begin{minipage}{0.32\textwidth}
	\begin{center}
    		\includegraphics[width=0.8
    		\textwidth]{university}
	\end{center}
	\end{minipage}
    \end{center}
\end{titlepage}
\newpage\null\thispagestyle{empty}\newpage


% ABSTRACT
\chapter*{Abstract}
\thispagestyle{empty}
The abstract and the keywords should not exceed the limit of this page
%\newline\newline
%\emph{Keywords:} Keywords should be written in order of relevance

% SYNOPSIS
\chapter*{Synopsis}
\thispagestyle{empty}

\paragraph{Background}
To mitigate the harmful effects of air pollution, air quality is regularly monitored. Such monitoring produces massive amounts of data, and this enables the development of statistical and machine learning techniques for modeling and forecasting of air pollution.
\paragraph{Problem}
Air pollution is a complex phenomenon depending on many factors, and the data from air monitoring has both temporal and spatial dependencies. This makes modeling and forecasting a challenge, and the research problem here is: \textit{To capture and model the complex dynamics of air pollution with machine learning methods, with an emphasis on deep learning}.

\paragraph{Research Question}
The research question for the thesis is: \textit{How can machine learning, in particular deep learning, be used to forecast air pollution levels and pollution peaks?} The research question emphasizes pollution peaks as these are much harder to accurately predict than pollution levels in lower ranges.  %Also, sudden jerky leaps, where pollution levels quickly rise tend to give strongly underestimated predictions, and this can have undesired consequences
%
%All type of forecasts are always imprecise, and generally for air pollution forecasts, the largest errors are seen during episodes with high pollution levels. Also, sudden jerky leaps, where pollution levels quickly rise tend to give strongly underestimated predictions, and this can have undesired consequences. The research question in this work are therefore as follows: 

\paragraph{Method}
Data was downloaded from SMHI's centralized database for air measurements, carefully examined, and thereafter preprocessed before a linear regression and several deep learning models were fit. All models were trained with historical data and later evaluated on the most recent (test) data. An important aspect of model evaluation was a close examination of the forecast errors. The Python programming language was used together with libraries for scientific programming and data science/machine learning.

\paragraph{Result}
All deep learning models outperformed the linear regression model. However, for all models, the structure of the forecast errors were not as desired, and this warrants further model refinements. The structure of the errors were due to the inability of the models to capture pollution peaks well. Though the deep learning models showed promising potential, in light of the research question, the results were unsatisfactory. 

\paragraph{Discussion}
The forecast horizons in this work (one hour) are very short-term, which limits usability. To this end however, adaptations to extend the forecast horizon are possible. Further adaptations to include more than just one air pollutant are also possible; this would result in very comprehensive forecasts that could be of great value to public health authorities and policy makers as they could permit early interventions in order to protect public health, and vulnerable groups in particular.

% ACKNOWLEDGEMENTS
\chapter*{Acknowledgements}
\thispagestyle{empty}

\tableofcontents
\thispagestyle{plain}
\thispagestyle{plain}
\setcounter{page}{1}
\pagenumbering{roman}

\listoffigures
%\setcounter{page}{1}
%\pagenumbering{roman}
\thispagestyle{plain}

\listoftables
\thispagestyle{plain}

\chapter*{List of Abbreviations}
\begin{tabular}{l@{$\dots\dots$}p{12cm}}
%Convolutional neural network\dotfill  & CNN \\
Environment and Health Administration \dotfill & EHA \\
Fully connected neural network\dotfill  & fcNN \\
Gated recurrent unit \dotfill & GRU \\
Long short-term memory \dotfill & LSTM \\
Mean absolute error\dotfill & MAE \\
Mean error \dotfill & ME \\
Mean squared error\dotfill & MSE \\
Multiple linear regression\dotfill & MLR \\
Ordinary least squares \dotfill & OLS \\
Recurrent neural network\dotfill & RNN \\
Root mean squared error\dotfill & RMSE\\
Particulate matter \dotfill & PM \\
Swedish Meteorological and Hydrological Institute\dotfill & SMHI \\
Stockholms Luft- och Bulleranalys \dotfill & SLB-analys \\
Volatile organic compounds \dotfill & VOCs \\
World Health Organization \dotfill & WHO \\
\end{tabular}

\addcontentsline{toc}{chapter}{List of Abbreviations}

\addtocontents{toc}{\bigskip}


\chapter{Introduction}
\setcounter{page}{1}
\pagenumbering{arabic}

\section{Background}

Outdoor air pollution is a major global environmental issue, linked to several serious health conditions, and causing millions of premature deaths every year \cite{who2016}. Some principal air pollutants damaging to health include gaseous substances such as nitrogen oxides (NO$_x$), ground-level ozone (O$_3$), sulphur dioxide (SO$_2$), and carbon monoxide (CO), but also atmospheric aerosol particles such as PM$_{10}$ and PM$_{2.5}$ \cite{VanLoon2010}. In Stockholm, traffic is a major source of local air pollution, and though air quality is generally good, some streets experience short episodes with severe pollution levels, especially during winter and spring \cite{slbanalys}. %The air in the Stockholm region is regularly monitored by the environmental and health administration (EHA) of the city of Stockholm \cite{slb-matningar}. 

To protect public health, urban air is normally monitored. In addition to monitoring, forecasts of air quality (both hourly and daily) can be critical to regulatory authorities, and in general, there are two approaches to this; with mechanistic models or statistical and/or machine learning models. \cite{ElHarbawi2013, Liao2020, atmos7020015}. %The deterministic dispersion models are common and used e.g. by the EHA of Stockholm \cite{slbanalys}. Some commonly used statistical models include multiple linear regression (MLR), autoregressive integrated moving average (ARIMA), and variants thereof \cite{atmos7020015}. 
With mechanistic models, the processes governing the evolution of air pollution is modeled mathematically, whereas statistical and machine learning models are more data-driven \cite{atmos7020015}. 

From a statistical perspective, predicting air pollution is a time series regression problem, and there are many different regression techniques for forecasting and time series analysis \cite{atmos7020015}. These techniques can vary in complexity, from more simple linear models to deep neural networks capable of finding complex non-linear relationships in the data \cite{atmos7020015, LeCun2015}. Nonetheless, one of the main challenges with air pollution is that there are dependencies over both space and time (i.e., the data is spatio-temporal), and simpler models may not capture these dependencies \cite{Liao2020}. \textcolor{red}{Recent advances in machine learning however have shown promising results when it comes to air quality forecasts, especially deep neural networks \cite{Liao2020, atmos7020015}.} 

%Some variants of combined deep neural networks with one spatial and one temporal component have been implemented however, and these "hybrid networks" have shown promising results \cite{Liao2020, Gilik2021}. 

%However, many common statistical and machine learning techniques have some limitations when applied to air pollution data. For example, they may not be ideally suited when there are dependencies over time, i.e, when the data is sequential (or so-called time-series data), and they may not capture spatial dependencies between different monitoring stations  \cite{Liao2020}. However, with the recent advances in deep learning \cite{LeCun2015}, many of the issues with large, high-dimensional "spatio-temporal" datasets have been  addressed \cite{Liao2020}. Deep learning is capable of detecting intricate input-output relationships in massive datasets, and there are variants of deep neural networks "tailored" to handling both sequential data as well as data where spatial dependencies are important \cite{LeCun2015}. For air pollution forecasts, variants of combined deep neural networks with one spatial and one temporal component have been implemented and these hybrid networks have shown promising results \cite{Liao2020, Gilik2021}. 

%More recently, machine learning methods have become increasingly popular, and some common algorithms for predicting air pollution are random forests, gradient boosting machines, and support vectors machines \cite{FaganeliPucer2018, Arsov2021}. Though successfully implemented, many of the statistical and machine learning models have some limitations when applied to air pollution data. They are not ideally suited when there are dependencies over time, i.e, when the data is sequential (or so-called time-series data), and they may not capture spatial dependencies between different monitoring stations  \cite{Liao2020}. Moreover, the relationship between predictor variables and the response variable(s) in air pollution data may also be very complex and non-linear, and often the data is high-dimensional \cite{Liao2020}. 

%With the recent advances in deep learning \cite{LeCun2015}, many of the issues with large, high-dimensional "spatio-temporal" datasets have been  addressed \cite{Hamdi2021}. Deep learning is capable of detecting intricate input-output relationships in massive datasets, and there are variants of deep neural networks "tailored" to handling both sequential data as well as data where spatial dependencies are important \cite{LeCun2015}. For air pollution forecasts, variants of combined deep neural networks with one spatial and one temporal component have been implemented and these hybrid networks have shown promising results \cite{Liao2020, Gilik2021, Qin2019}. 

%In this work, historical data from several air monitoring stations in the Stockholm region will be used together with weather data to make prediction models of various complexity. The performance of the models will be evaluated and compared to similar work in the literature, but also to the models used by the EHA of Stockholm (mechanistic models). Lastly, the models will be tested on data from other cities, and a transfer learning approach will be used.

\section{Research problem}

Forecasts, be it for weather, stock returns, or future pandemics, are always associated with uncertainty and errors. Erroneous predictions made by existing air pollution forecasting systems, both mechanistic and statistical and/or machine learning-based, can be attributed to many causes. 
In the case of mechanistic models, there can be insufficient information in terms of the factors needed for simulation and modeling \cite{atmos7020015}. For statistical and/or machine learning methods, too simplistic models, lack of data, irrelevant input features, overfitting, etc., can limit prediction accuracy \cite{atmos7020015}. Nevertheless, atmospheric pollution is a very complex phenomenon depending on a multitude of factors across both space and time. Hence, the research problem addressed in this work is:
%\begin{quotation}
%\noindent
\textit{To capture and model the complex dynamics of air pollution with modern machine learning methods, with an emphasis on deep learning.}
%\end{quotation}

%\textit{using state-of-the-art machine learning methods, in particular deep learning, to more successfully capture and model the complex dynamics of air pollution.}

%the research problem this work attempts to address is if state of the art machine learning solutions successfully can model these complex dynamics of air pollution. 

%and existing forecasting models/systems for air pollution make erroneous predictions which tend to be largest when pollution levels peak. In the case of mechanistic models, this is due to insufficient information about necessary parameters  and 

\section{Research question}
From a forecasting perspective, of special interest are episodes when pollution levels peak. Generally, this is also when existing forecasting systems tend to give the largest prediction errors \cite{atmos7020015}. Therefore, the research question this thesis tries to answer is:
%\begin{quotation}
%\noindent
\textit{How can machine learning, in particular deep learning, be used to forecast air pollution levels and pollution peaks?}
%\end{quotation}

%\begin{itemize}
%\item How can machine learning, in particular deep learning, be used to forecast air pollution levels and pollution peaks?
%\makebox[\textwidth]{\textit{How can machine learning, in particular deep learning, be used to forecast air pollution levels and pollution peaks?}}
%\end{itemize}

%an additional sub-question related to the main research question is: \textit{Can machine learning models reliably predict peaks in pollution levels?}


%\begin{center} \textit{Can machine learning models reliably predict peaks in pollution levels?}  \end{center}

%How do machine learning, in particular deep learning, perform in comparison with the dispersion models used by the EHA of Stockholm? Can machine learning algorithms trained on Stockholm data be successfully transferred to other cities with similar conditions? Can density predictions instead of point-predictions be constructed so that if thresholds are of interest, the probability of air pollution levels exceeding a certain level can be calculated?  

\section{Delimitations}

%In this work, historical air pollution and weather data is used. Therefore, the models cannot be tested in "operational mode", i.e., with real-time data to make predictions. Moreover, when forecasting air pollution (hours or days ahead), weather forecasts are often utilized in addition to monitoring data to improve forecast results. Again, with historical data, incorporating weather forecasts is not possible, and consequently the models make predictions based only on multivariate time-series of past observations. This also puts a limit on the time-horizon for the predictions, since without weather forecasts, medium to long-term forecasts of air quality would have large uncertainties.  

%\textcolor{red}{pollution background levels...}


\chapter{Extended Background}
\section{Ambient air pollution}

Ambient air pollution is one of the greatest environmental and health concerns of the modern world. Worldwide, poor air quality causes millions of premature deaths every year and is linked to several adverse health effects such as respiratory problems, cardiovascular disease, and cancer \cite{who2016}. In addition to health risks, the global economic impacts are substantial due to lost labor productivity, increased health care costs, reduced crop yields, etc.\ \cite{oecd2016}. Outdoor air pollution has become a ubiquitous problem, affecting both cities and rural areas, and it is estimated that about 90\% of the world's population are living in regions where air pollution levels exceed guidelines set by the World Health Organization \cite{who2016}. 

\subsection{Principal air pollutants}
\label{sec:airpollutants}
In densely populated urban areas, air pollution levels can periodically be severe, and with an accelerating urbanization, it has become imperative for regulatory authorities to closely monitor city air and try to mitigate the harmful effects of pollution. Commonly monitored substances include sulphur dioxide (SO$_2$), nitrogen oxides (NO$_x$, i.e., NO and NO$_2$), carbon monoxide (CO), ground-level ozone (O$_3$), volatile organic compounds (VOCs), and particulate matter (PM) \cite{VanLoon2010}. Vehicular traffic is a major source of the gaseous pollutants NO$_x$, SO$_2$, CO, and VOCs, but certain industrial processes also contribute to emissions \cite{VanLoon2010}. Ground-level O$_3$ (also a gas) is a so-called secondary pollutant that forms when NO$_x$ and VOCs react on sunny days with little wind \cite{VanLoon2010}. 

PM, the group of pollutants being the focus of this work, are atmospheric aerosol particles (i.e., particles suspended in the air). They have diverse origins, both natural and anthropogenic, and a complex chemical composition consisting of both solid and liquid species \cite{Schwarzenbach2016}. Some important sources of PM are forest fires, volcanic eruptions, sand/dust storms, sea spray, vehicular traffic, certain industrial processes, construction sites, and domestic combustion \cite{Querol2004, Schwarzenbach2016}. When entering the atmosphere directly by these routes, one denotes the PM as primary. However, PM can also be formed by the oxidation of gases such as SO$_2$, NO$_x$, and VOCs (followed by a complex chemical reaction process), in which case the PM is said to be secondary \cite{Schwarzenbach2016}. PM is also categorized by particle size (or more specifically, the aerodynamic diameter), and particles measuring smaller than 2.5 \textmugreek m and 10 \textmugreek m are denoted as PM$_{2.5}$ and PM$_{10}$, respectively \cite{Schwarzenbach2016}.

Both PM$_{10}$ and PM$_{2.5}$ can travel long distances from point sources (though PM$_{2.5}$ has a longer residence time in the atmosphere than PM$_{10}$), and local pollution can be affected by regional background levels \cite{Schwarzenbach2016, slbanalys}. PM levels are also dependent on weather conditions \cite{Schwarzenbach2016}. For example, temperature and solar radiation are related to the formation of secondary PM, and PM emission from roads, tires, brake wear, etc., can be affected by precipitation and humidity \cite{slbanalys, atmos7020015}. Both PM$_{10}$ and PM$_{2.5}$ are hazardous and cause a wide range of health problems, though PM$_{2.5}$ can more easily penetrate the lungs \cite{Schwarzenbach2016}. In the European Union, annual mean limits are set to 40 \textmugreek g/m$^3$ for PM$_{10}$ and 20 \textmugreek g/m$^3$ for PM$_{2.5}$ \cite{eu-airquality}. 

\subsection{Ambient air pollution in Stockholm}
\label{air-pollution-stockholm}

In the city of Stockholm, environmental air quality standards are usually met, though some streets experience occasional episodes with severe pollution levels (e.g. Hornsgatan is one such street) \cite{slbanalys2021}. Since Stockholm has centralized district heating and few industries, the major source of local CO, NO$_x$, and PM pollution is vehicular traffic \cite{slbanalys, slbanalys2021}. Mechanical wear by studded tires on asphalt and the wearing of brakes and tiers in motor vehicles contribute substantially to local levels of both PM$_{10}$ and PM$_{2.5}$. For PM$_{2.5}$ however, contribution from sources outside of Stockholm is also significant \cite{slbanalys2021}. Emission of SO$_2$ can come from the energy sector and waterborne transport, though local levels are also affected by outside sources. 
%The levels of SO$_2$ are affected by transport from outside sources, though local and regional emissions can be due to the energy sector and waterborne transport \cite{slbanalys2021}. 
For O$_3$, long-range transport from mainland Europe is the single-most important factor contributing to locally measured levels \cite{slbanalys2021}. 

The air in Stockholm County is monitored by Stockholms Luft- och Bulleranalys (SLB-analys), a unit in the Environment and Health Administration (EHA) of the city of Stockholm. SLB-analys are responsible for a number of monitoring stations measuring several air pollutants and some meteorological parameters in the Stockholm region, as well as a few stations outside of Stockholm \cite{slb-matningar}. In addition to monitoring the air, SLB-analys also model and forecast air pollution levels for the Stockholm metropolitan area, and their forecasts are available through a smartphone application, called "Luft i Stockholm" \cite{slbanalys}. 

\section{Forecasting air pollution}

Having the possibility to forecast air pollution levels hours or days ahead can be extremely valuable to regulatory authorities in order to protect public health, and vulnerable groups in particular. In general, there are two broad categories of models for such forecasts; mechanistic models, and statistical and/or machine learning models \cite{ElHarbawi2013}. This work is concerned with the latter type, and in the sections below a review follows. The mathematical and statistical theory behind many of the models is quite extensive \cite{Hastie2009, Montgomery2015, smlbook, LeCun2015}, but relevant theory will be covered briefly.

\subsection{Forecasting as a regression problem}
\label{sec:forecasting}
While mechanistic models are based on mathematical modelling of atmospheric processes along with other factors governing the distribution of air pollution (such as emission source characteristics, physico-chemical properties of pollutants, terrain and building design, etc.) statistical and/or machine learning models are entirely data-driven, being derived directly from measurements on the variables of interest \cite{ElHarbawi2013}. From a statistical (or machine learning) perspective, forecasting air pollution can be viewed as a regression problem, in which a function $f$, mapping input data to a numerical output, is being approximated (or learned) from a training set of labeled input-output examples \cite{smlbook}. Learning the function $f$ amounts to finding a set of parameters (or weights) for the model, which in the case of a simpler regression technique can be only a handful, but possibly millions if a deep neural network is used \cite{smlbook}. Generally in regression, the weights are learned by minimizing a cost function
\begin{align}
J(\bm{\hat{\beta}}) = \frac{1}{n} \sum_{i=1}^{n} \big(\hat{y}_i - y_i \big)^2
\label{eq:cost}
\end{align}
where $\bm{\hat{\beta}}$ is the vector of estimated model parameters ($\hat{\beta}_0, \: \hat{\beta}_1, ..., \: \hat{\beta}_n$), $\hat{y}_i$ is a prediction and $y_i$ is a training data value\cite{smlbook}. In \cref{eq:cost} the squared error loss is used as loss function, and the cost is simply the loss averaged over the training data.\footnote{What is meant by cost and loss functions can vary slightly in the literature, but in this work, the terminology of Lindholm et al. \cite{smlbook} is adopted.} Depending on the model, minimizing $J(\bm{\hat{\beta}})$ is approached differently, as explained further in the sections below. 

\subsection{Multiple linear regression models}
%\paragraph{Multiple linear regression}
From the wealth of available regression techniques, multiple linear regression (MLR) has been extensively used to forecast and model air pollution \cite{atmos7020015}. Generally, if none of the basic model assumptions are violated (i.e., linearity, independence, normality, and constant variance), MLR is often a straightforward method, especially for data with no temporal dependencies (so-called cross-sectional data). However, for time series data, the assumption of independent errors is often not approptiate \cite{Montgomery2015}. 

If fitting a MLR model to time series data, successive errors will typically be correlated (often referred to as autocorrelation), and this will cause several problems with the model if the correlation is not accounted for \cite{Montgomery2015}. To this end, adjustments to the MLR model can be made, some of which require other parameter estimation techniques than the usual least squares method (see below). However, a simple and commonly used procedure to get rid of the autocorrelation is to include one or more lagged values of the response variable as predictors. For example, if the value of the response variable at lag one ($y_{t-1}$) is included, the MLR model will have the form 
\begin{align}
y_t = \beta_0 + \beta_{1} y_{t-1} + \beta_2 x_{2,t} + ... + \beta_{k} x_{k,t} + \varepsilon_t, \: \: \: t = 1, 2, ..., T
\label{eq:mlr}
\end{align}
where $\varepsilon_t$ is the the error term, and $t$ denotes time steps \cite{Montgomery2015}. The model in \cref{eq:mlr} can be fit with the method of least squares, which in linear regression is the standard way of finding parameters so that $J(\bm{\hat{\beta}})$ is minimized \cite{Montgomery2012}. This is done by solving the so-called normal equations,
%\begin{align}
%(\bm{X}^T\bm{X})\bm{\hat{\beta}} = \bm{X}^T\bm{y}
%\label{eq:normal_eq}
%\end{align}
and the least squares estimates of the model parameters are then given by
%(provided that the inverse of $\bm{X}^T\bm{X}$ exists) 
%are given by \cref{eq:normal_sol} below.
\begin{align}
\bm{\hat{\beta}} = (\bm{X}^T\bm{X})^{-1}\bm{X}^T\bm{y}.
\label{eq:normal_sol} 
\end{align}

Careful variable selection in regression is crucial as it can influence the performance of a model, and one is often concerned with finding an optimal ”subset” of predictors, where multicollinearity should also not be an issue \cite{Montgomery2012}. To this end, variable selection techniques based on optimizing a criterion like the Akaike or Bayes information criterion are common, and typically multicollinearity is also tested for \cite{Montgomery2012}. However, if one is reluctant to exclude variables, but multicollinearity still might be an issue, regularized versions of MLR can be used \cite{Montgomery2012}. Two common techniques are $L_1$ and  $L_2$ regularization, in which an extra so-called "penalty" term is added to the cost function to shrink (and essentially stabilize) the model parameter estimates \cite{smlbook}. In $L_2$ regularization (also called ridge regression), the parameters will be pushed towards small values, whereas in $L_1$ regularization (or lasso regression), some parameters will be driven to zero. 
%The penalty terms for ridge and lasso regression are, respectively, 
%$$\lambda \sum_{j=1}^{k}\beta_{j}^2\:\:\: \text{and}\:\:\: \lambda \sum_{j=1}^{k}|\beta_{j}|$$
%where $\lambda$ is a parameter controlling the shrinkage \cite{smlbook}. For ridge regression, the parameter estimates can be found by solving a modified version of \vref{eq:normal_sol}, while for lasso regression, no such analytical solution exists, and numerical optimization techniques have to be utilized \cite{smlbook}. 
Generally, $L_1$ and $L_2$ regularization can be used to prevent overfitting, and also for $L_1$ where some input variables can be eliminated, as a variable selection technique \cite{smlbook}. 

The extensive use of MLR for air pollution forecasts is many times motivated by its simplicity and straightforward implementation \cite{atmos7020015}. Another advantage is interpretability; for example, inference can be made on all input variables, allowing one to investigate their individual importance \cite{Montgomery2012}. However, the assumption of linearity might not always hold, and large prediction errors have been observed at times of pollution peaks \cite{atmos7020015}. Moreover, with data from several (but nearby) monitoring stations, multicollinearity among the input variables can be an issue, which is why ridge or lasso regression are popular alternatives to the non-regularized MLR model \cite{atmos7020015}. 

%Linear models for time-series analysis, such as auto-regressive moving average and auto-regressive \textit{integrated} moving average (ARMA and ARIMA, respectively) and variants thereof, are also common \cite{Arsov2021, Goyal2006}. These models make predictions of future values based on past data (i.e., taking dependencies over time into account), however, similar to MLR, linearity is assumed and errors can be large when there are temporary peaks in pollution levels \cite{atmos7020015}. 

%\paragraph{Bayesian methods}
%While the forecasting techniques discussed above produce point-predictions of a pollutant, Bayesian methods can be used to predict distributions (or put another way, make density forecasts) \cite{FaganeliPucer2018}. A Bayesian approach can offer some advantages if thresholds are of interest, since with a density forecast, the probability of pollution levels exceeding a certain value can be estimated \cite{FaganeliPucer2018, smlbook}. For example, in Pucer et al. \cite{FaganeliPucer2018}, a Gaussian process model was used to give Gaussian density predictions of PM$_{10}$ and O$_3$. Bayesian methods have also been used for spatial predictions of PM by spatial interpolation (using values from monitored locations to estimate levels at other locations without any monitoring) \cite{atmos7020015}. 

\subsection{Extensions of the linear model}
More versatile and flexible regression models tend to give better forecasting results than linear models \cite{atmos7020015}. Some examples include regression trees, generalized additive models, and support vector machines (SVM) \cite{atmos7020015, FaganeliPucer2018}. These models can handle more complex non-linear input-output relationships, and especially SVM has been successfully applied for PM$_{10}$ prediction, sometimes with better results than artificial neural networks \cite{atmos7020015}. 

Artificial neural networks (ANNs), in particular the multilayer perceptron (MLP), have also been extensively used as a forecasting technique \cite{atmos7020015}. ANNs are flexible models able to handle non-linear input-output relationships, however, over-fitting can be an issue, especially with high-dimensional input and if training data is limited \cite{atmos7020015, FaganeliPucer2018}. 

%The MLP is a so-called feedforward neural network, in which a set of inputs are taken, passed through several layers of so-called hidden units, eventually producing an output \cite{LeCun2015}. 
The MLP is a so-called feedforward neural network, in which a set of input data is taken and passed through several "hidden" layers made up of neurons (also called units), before an output is produced \cite{LeCun2015}.
%A common way to illustrate a MLP is given in Figure \ref{fig:ANN}, where a network consisting of the input layer, two hidden layers (where units are represented with circles), and a single output, is shown. 
Deep neural networks can have many such layers (hence the term "deep" \cite{Chollet2017}), and each layer can have hundreds of units. Every layer produces a slightly more abstract representation of its input by non-linear transformations, and with several such transformations, complex relationships in the data can be learned \cite{LeCun2015}.

%\begin{figure}[h]
%\begin{center}
%\includegraphics{neural-network}
%\caption{Artificial neural network with two hidden layers.}
%\label{fig:ANN}
%\end{center}
%\end{figure}

%Training the neural network is an iterative process, in which the weights (or parameters) of the network are adjusted until the measured error stops decreasing.

Many other deep learning architectures than the MLP exist, such as convolutional neural networks (CNNs), or recurrent neural networks (RNNs). CNNs are commonly used for image recognition while RNNs (and variants thereof) normally are applied to sequential data. 
%\textcolor{red}{(This section will be expanded with some more theory for the deep learning models to be used. Also, some mathematical notation will added.)}

%\subsection{Variable selection}
%
%In any regression problem, variable/feature selection is crucial as it can influence the performance of a model. 
%%, and rarely are all available input features necessary or even desirable to include (as some might worsen performance) \cite{smlbook}. 
%%As pointed out in section \ref{sec:airpollutants}, 
%With regards to PM, as pointed out in section \ref{sec:airpollutants}, weather conditions can greatly affect pollution levels, and therefore meteorological data can be utilized to improve forecasts \cite{atmos7020015}.
%
%% ### CORRELATION PLOT ###
%\begin{figure}[h]
%\makebox[\textwidth][c]{\includegraphics[width=0.85\textwidth]{/Users/simoncarlen/desktop/luftdata/plots/Torkel Knutssongatan_correlation}}
%\caption{Pairwise correlations between air pollutants and some meteorological variables.}
%\label{fig:correlationplot}
%\end{figure}
%% #######################
%
%Additional variables can also be included in PM forecasts. For example, motor traffic data such as travel speeds, traffic flow and intensity, etc., can be utilized \cite{atmos7020015}. Data on other pollutants can also be important, especially SO$_2$ and NO$_x$ as they are involved in the formation of secondary PM \cite{Arsov2021}. Moreover, if forecasts focus solely on PM$_{10}$ (as in this work), data on PM$_{2.5}$ can further improve the results  \cite{Arsov2021}. Temporal variables such as time of the day and time of year are also useful since daily and seasonal variation of PM pollution is important \cite{Schwarzenbach2016, atmos7020015}. 
%%The selection and preprocessing of variables in this work is described in detail in section \ref{chap:dataprocesschap}. 
%%A detailed description of all variables utilized in this work (and how they were processed), is given in section \ref{chap:dataprocesschap}.
%
%In Figure \ref{fig:correlationplot} where pairwise correlations between a few meteorological variables and PM$_{10}$ at different stations in the Stockholm region are given (see Table \ref{tab:stations} for details about the different monitoring stations), it can be inferred that PM$_{10}$ correlate negatively with humidity, but positively with atmospheric pressure and solar radiation. It can also be seen that PM$_{10}$ levels are strongly correlated among some stations. 
%
%In this work, in addition to PM$_{10}$ data, meteorological data as well as data on PM$_{2.5}$ and NO$_x$ were utilized. Some features used as input to the models were also derived. A more detailed description of the variables and their preprocessing is given in section \ref{chap:dataprocesschap}.

\section{Summary and motivation for this work}

\chapter{Methodology}
%General implementation of the stategy here before going into details about sources, preprocessing, hyperparameter tuning etc, ... Perhaps also a pic?
%The major steps of the implemented workflow were as follows; 
%
%Detailed descriptions of each step in the process are given in subsequent sections

%is shown in Figure \ref{fig:dataflow}. Historical air pollution data from several monitoring stations, together with meteorological data from one station, was retrieved, preprocessed (with some features engineered), and divided into data windows. Three deep learning models (feed forward neural network, RNN, and LSTM) were trained and tested for short-term predictions (one hour ahead) of PM$_{10}$ for one station at Torkel Knutssongatan (measuring urban background levels, see Table \ref{tab:stations}). As baseline models for comparison, multiple linear regression and ARIMA were used. Detailed descriptions of each step in the process are given in subsequent sections. 

%\begin{figure}[htbp]
%\begin{center}
%\makebox[\textwidth][c]{\includegraphics[width=1\textwidth]{workflow}}
%\caption{Implemented workflow.}
%\label{fig:dataflow}
%\end{center}
%\end{figure}

\section{Data retrieval and preprocessing}
\label{chap:dataprocesschap}

\subsection{Data sources}
\label{sec:data-sources}

Air pollution data was retrieved from the Swedish Meteorological and Hydrological Institute's (SMHI) centralized database for air quality measurements \cite{smhi-luftmatningar}. This data is part of the national and regional environmental monitoring of Sweden, a program coordinated and funded by the Swedish Environmental Protection Agency (Swedish EPA) and the Swedish Agency for Marine and Water Management. There are in total ten different program areas, of which air is one, and all data are licensed under CC0 and therefore freely accessible to the public \cite{naturvardsverket-miljodata}. For the national air monitoring (under Swedish EPA's responsibility), SMHI acts as a national data host and stores (quality checked) historical data reported yearly from municipalities in Sweden \cite{smhi-luftmatningar}.

\subsubsection{Monitoring stations}
In Stockholm County, there are 19 stationary sites for air pollution monitoring \cite{slb-matningar}, and initially, data from each was considered. However, not all sites measure hourly NO$_2$, and for some sites the data were irregular. Therefore, data from four sites with hourly NO$_2$ measurements (in \textmugreek g/m$^3$) for the time period 2016-01-01 to 2022-01-01 was subsequently chosen, giving a total of 52,609 data points. (However, as explained further below, a year worth of data, i.e., 8760 data points had to be excluded.) For the station at which NO$_2$ predictions subsequently were to be made (Torkel Knutssonsgatan), hourly meteorological data was also utilized. More specifically, these meteorological variables were temperature (in $^\circ$C), precipitation (mm), atmospheric pressure (hPa), relative humidity (as \%), solar radiation (W/m$^2$), and wind speed (m/s). The meteorological data were downloaded from SLB-analys' webpage \cite{slb-analys-meteorologi}. 

In general, air pollution monitoring can be classified by the surrounding area (rural, rural-regional, rural-remote, suburban, and urban), and by the predominant emission sources (background, industrial, or traffic) \cite{smhi-luftmatningar}. The chosen stations included data from both traffic and background monitoring, in urban as well as rural-regional areas. More information about the stations are given in Table \ref{tab:stations} in appendix \ref{chapt:appendix_A}. 

%% STATIONS TABLE
%% ##############
%\begin{table}[]
%\centering
%\caption{Monitoring stations in Stockholm County.}
%\label{tab:stations}
%\resizebox{\textwidth}{!}{%
%\begin{tabular}{@{}llllll@{}}
%\toprule
%Station                         & Station code & Longitude & Latitude  & Type of monitoring                                                   & Parameters                                                                                             \\ \midrule
%Norrtälje, Norr Malma           & 18643        & 18.631313 & 59.832382 & \begin{tabular}[c]{@{}l@{}}Rural-Regional \\ Background\end{tabular} & PM$_{10}$, PM$_{2.5}$                                                                                  \\ \midrule
%Stockholm, Hornsgatan 108       & 8780         & 18.04866  & 59.317223 & Urban Traffic                                                        & PM$_{10}$, PM$_{2.5}$                                                                                  \\ \midrule
%Stockholm, Torkel Knutssonsgatan & 8781         & 18.057808 & 59.316006 & Urban background                                                     & \begin{tabular}[c]{@{}l@{}}PM$_{10}$, PM$_{2.5}$, NO$_2$, \\ meteorological \\ parameters\end{tabular} \\ \bottomrule
%\end{tabular}%
%}
%\end{table}
%% ##############

\subsection{Data preprocessing}
\subsubsection{Initial preprocessig}
%(Similar plots for all stations are given in Fig.\ \ref{fig:time_series_plots_all} in appendix \ref{chapt:appendix_A}.)
%(station-wise, with PM$_{10}$ and PM$_{2.5}$ in the left and right subplots, respectively)
All stations had short episodes with missing data, and linear interpolation was used to fill in the missing values. Missing weather data was also linearly interpolated, except atmospheric pressure and wind speed for which mean imputation was deemed more appropriate. Moreover, before use in any of the models, all data were min-max normalized (i.e., scaled to the interval $[0,1]$). 
%However, for the PM$_{2.5}$ data at Torkel Knutssonsgatan (plot (b) in \vref{fig:time_series_plots}), due to the rather big gap at the beginning of 2019,

In \vref{fig:time_series_plots}, the NO$_2$ data for Torkel Knutssonsgatan is shown. A notable reduction in NO$_2$ levels can be seen during 2020 and early 2021; this reduction is most likely due to the COVID-19 pandemic, and by late 2021, pre-pandemic NO$_2$ levels are again approached.  Because of this, a train-test split (see below) was done to entirely avoid using the data from 2020.
% (which was an unusual year with regards to air pollution levels). 
%a train-test split was done to entirely avoid this period (see below). Missing weather data was also linearly interpolated, except for the variables atmospheric pressure and wind speed for which mean imputation was deemed more appropriate. 

%It should be noted that some PM values were negative.
%%this is seen clearly in e.g.\ plot (f) in \vref{fig:time_series_plots}, for the time period Jan.\ 2016 up to about Sep.\ 2019. 
%However, negative values are expected since automated measuring instruments for PM (due to noise) may produce values between zero and the negative detection limit, especially when there are rapid changes in humidity (SMHI, personal communication, April 11, 2022). These values are therefore not to be considered any more "incorrect" than positive values, though it may at first seem contradictory to include them in an analysis. 

\begin{figure}[h]
\centering
\makebox[\textwidth][c]{\includegraphics[width=1\textwidth]{../plots/NO2_Torkel_Knutssonsgatan}}
%\includegraphics[width=\textwidth]{../plots/time_series_plots}
%\caption{Time series plots for (a) PM$_{10}$ and (b) PM$_{2.5}$ at Torkel Knutssonsgatan.}
\caption{NO$_2$ data for Torkel Knutssonsgatan.}
\label{fig:time_series_plots}
\end{figure}

%\begin{figure}[h]
%\centering
%\makebox[\textwidth][c]{\includegraphics[width=0.9\textwidth]{../plots/time_series_plots_7}}
%\caption{Time series plots for PM$_{10}$ and PM$_{2.5}$.}
%\label{fig:time_series_plots}
%\end{figure}

%\paragraph{Feature engineering}
%From the meteorological data, wind vectors ($u$ and $v$) were derived from wind direction and wind speed, as wind vectors are more suitable model inputs \cite{tensorflow-timeseries}. After converting wind direction values to radians, $u$ and $v$ were obtained in the following way $$ u = ws * cos(\theta)$$ $$v = ws * sin(\theta)$$ where $ws$ denote wind speed and $\theta$ is the wind direction (in radians). 
\subsubsection{Creating temporal variables}
In \cref{fig:time_series_plots}, yearly periodicity in the data can be seen, where levels tend to peak during winter months. Daily and weekly periodicity is also expected since traffic intensities vary throughout the day and week. 
%The meteorological variables such as temperature, solar radiation, etc.\ also have periodicity. 
To account for this, timestamps were converted to temporal variables as sine and cosine waves for day, week, and year. For example, the sine and cosine waves for day were calculated in the following way

$$ \text{Sine day} = \frac{1}{2} \Big( \text{sin} \Big (\text{timestamp} \cdot \frac{2\pi}{86,400} \Big) + 1 \Big)$$
$$ \text{Cosine day} = \frac{1}{2} \Big( \text{cos} \Big (\text{timestamp} \cdot \frac{2\pi}{86,400} \Big) + 1 \Big)$$
where timestamp is in UNIX epoch time\footnote{The UNIX epoch time for a given timestamp $t$ is the number of seconds that has passed between January 1, 1970, and $t$.} (and with 86,400 seconds in 24 hours, dividing by this term is necessary). The calculations were done similarly for week and year, except for the term in the denominator which instead was set to seconds per week and seconds per year, respectively. Note that the sine and cosine waves were adjusted to oscillate between zero and one. The temporal variables for day in a 24 hour time window are shown in \vref{fig:time_sine_cos}.

\begin{figure}[h] 
\begin{center}
%\makebox[\textwidth][c]{\includegraphics[width=.585\textwidth]{../plots/time_signals}}
\includegraphics[scale=1.05]{../plots/time_signals}
\caption{Temporal variables for day as sine and cosine waves.}
\label{fig:time_sine_cos}
\end{center}
\end{figure}

\subsubsection{Rolling windows}
The rolling windows method extracts data sequences of certain lengths (the "windows") from the input data, and in each window is an "input window" and a "target window" \cite{Gilik2021}. For example, as shown in Figure \ref{fig:sliding-window}  (where $t$ indicate time steps), with a sequence of nine data points, the first eight observations would constitute the input window, and the ninth observation the target window. After extracting this sequence, a slide forward is made to extract the next sequence, and this is continued until observation $n$ becomes the target window, at which point all the data have been processed.% and divided into smaller windows. 

In this work, rolling windows were used as input to the deep learning models, and different input window lengths were tested for making predictions of a target window one time step ahead (i.e., the next hour). More details are given in section \ref{sec:tuning_deep_learning} below.
% for short-term predictions of a target window one step ahead (i.e., the next hour). More details are given in section \ref{sec:tuning_deep_learning} below. 
 \begin{figure}[h]
\begin{center}
%\includegraphics{sliding-windows}
\makebox[\textwidth][c]{\includegraphics[width=1\textwidth]{sliding-windows}}
\caption{Rolling window approach for time-series data.}
\label{fig:sliding-window}
\end{center}
\end{figure}
%In this work, input window lengths were set to 12 hr, and predictions were made 1 hr, 6 hr, and 12 hr in the future (giving total window lengths of 13 hr, 18 hr, and 24 hr, respectively). Moreover, single-output models predicting PM levels at one urban background station (Torkel Knutssongatan),  as well as multi-output models predicting PM levels at all urban traffic stations, were tested.
%In this work, the window lengths were set to 12 hours, and both single time-step predictions (target windows of length one) and multi-time-step predictions (target windows of length $n$) were tested. Moreover, prediction models giving both single-outputs (the target value at one monitoring station) as well as multi-outputs (target value at all stations) were constructed.
%\paragraph{Train-test split} Lastly, the data was split into training (60\%), validation (20\%), and test (20\%) sets, where the validation set was used for hyperparameter tuning (described in more detail in section \ref{sec:tuning}). Being sequence data, the sampling was done consecutively, without random shuffling, so that order information would be preserved \cite{Gilik2021}. 
\subsubsection{Train-test split} 
Lastly, the data was split into training, validation, and test sets, where the validation set was used for hyperparameter tuning. The test set was taken as the most recent year of data (from 2021-01-25 to 2022-01-01, where the first 24 days of January were skipped due to many missing values at the Lilla Essingen station). For the validation set, the data from 2019 was used (since 2020 was an unusual year with regards to air pollution levels). The remaining data was used for training (2016-01-01 to 2019-01-01). This ordered (as opposed to random) split is motivated by the time dependence in the data. It should also be noted that when normalizing the validation and test sets, min and max values from the training set were used. This ensures that model evaluation will be a good (and not too optimistic) measure of how well the models generalize to new, previously unseen, data points.
%Being sequence data, the sampling was done consecutively, without random shuffling, so that order information would be preserved 

\section{Model fitting and hyperparameter tuning}

\subsection{Multiple linear regression models}
Initially, a simple linear regression model was fit with OLS where the response variable at lag one was used as predictor. No significant autocorrelation was seen with this model, but when also including the response variable at lag two as predictor, the Durbin-Watson statistic improved (i.e., was brought closer to 2). Including additional response variables after the first two lags did not lead to further improvements in terms of eliminating autocorrelation.
% however, the response variables at lag 24 and 25 turned out to also be highly significant. 

NO$_2$ data from other stations, meteorological variables, and the temporal variables were subsequently added to the model. These extra predictors did not lead to any serious multicollinearity, as indicated by the condition number. The NO$_2$ data was fit with the values at lag one, since for a forecast at time $t+1$, these predictors cannot be known. However, lagged values of the meteorological variables were not used as these can more easily be replaced with their forecasted values. A similar MLR approach to the one taken here, but for predicting daily means of PM$_{10}$, can be found Stadlober et al. \cite{Stadlober2008}. 

A log transformation of all NO$_2$ data was required before normalization to stabilize the variance of the errors, and also make the error distribution more normal. Even so, deviation from normality was indicated, and as can be seen in \cref{fig:residuals_MLR} in Appendix \ref{chapt:appendix_B} where OLS model diagnostics are shown, the error distribution had long/heavy tails. For this reason, a robust regression model with $M$-estimation (and Huber's $t$ function) was judged to be a more suitable alternative to OLS regression.
%as can be seen in the residual plots in \cref{fig:residuals_MLR} in Appendix \ref{chapt:appendix_B}. More specifically, the error distribution had long tails, which is why robust regression with $M$-estimates (and Huber's function) was judged to be a more suitable alternative to OLS regression. 

At this point, with many input variables in the model, recursive feature elimination (RFE) was used as a variable selection technique, in which the model is repeatedly re-fit after having removed the least significant predictor \cite{Faraway2020}. Each candidate model generated by RFE was evaluated on the validation set (as well as the training set for comparison), and the results are shown in \vref{fig:RFE}. 
\begin{figure}[h] 
\begin{center}
\includegraphics[scale=1.05]{../plots/RMSE_predictors}
\caption{RMSE for each candidate model generated by RFE.}
\label{fig:RFE}
\end{center}
\end{figure}

From \cref{fig:RFE}, it is evident that the least important predictors brought essentially no improvements to the model, and they were therefore removed. More specifically, these variables were; precipitation, atmospheric pressure, the Norr Malma NO$_2$ data, and the temporal variables for week and year. The final MLR model thus had the form
\begin{align}
%\log y_t = \beta_0 + \sum_{\mathclap{\substack{i=1,\\ l\in\{1,2,23,24\}\\}}}^{4} \beta_{i} \log y_{i,t-l} + \sum_{i=5}^{7} \beta_{i} \log x_{i,t-1} + \sum_{i=8}^{11} \beta_{i} z_{i,t}  + \sum_{i=12}^{13} \beta_{i} w_{i,t} +  \varepsilon_t
\log y_t = \beta_0 + \sum_{i=1}^{2} \beta_{i} \log y_{i,t-i} + \sum_{i=3}^{5} \beta_{i} \log x_{i,t-1} + \sum_{i=6}^{9} \beta_{i} z_{i,t}  + \sum_{i=10}^{11} \beta_{i} w_{i,t} +  \varepsilon_t
\label{eq:mlr}
\end{align}
where $x$ denotes input variables with NO$_2$ data from other than the target station, $z$ meteorological variables, $w$ the temporal variables for day, and t = 1, 2, ..., T. Also, the errors ($\varepsilon_t$) in \cref{eq:mlr} are assumed to follow a Gaussian-shaped, heavy-tailed probability distribution (see \cref{fig:residuals_MLR}b in Appendix \ref{chapt:appendix_B}). Summary statistics for this robust regression model, together with values and inference for the estimated parameters, are given in \cref{tab:Robust_table} in Appendix \ref{chapt:appendix_B}. Also, in \cref{tab:OLS_table} in Appendix \ref{chapt:appendix_B}, summary statistics for the OLS regression model are given, though this model was not used to make any forecasts. \\

%\begin{equation}
%\sum_{\mathclap{\substack{i=1,\\ j\in\{1,2,23,24\}\\}}}^{4}
%\end{equation}
%$$
%\sum_{(i,l) \in {(1,1), (2,2), (3, 23), (4,24)}} \betai y_{i, t-l}
%$\sum_{(i,l) \in\{(1,1), (2,2), (3,23), (4,24)\}}  \beta_{i} \log y_{i,t-l} $
%$$
%However, with many input variables in the model, an $L_1$-regularized robust regression was tested. 

%(Summary statistics for the OLS regression is given in Table \ref{tab:OLS_table} in Appendix \ref{chapt:appendix_B}.)

%It should be noted that a log transformation of all NO$_2$ data was required to stabilize the variance of the errors, and also bring them closer to a normal distribution. Even so, deviation from normality was indicated, as can be seen in the residual plots in \cref{fig:residuals_MLR} in Appendix \ref{chapt:appendix_B}. More specifically, the error distribution had long tails, which is why robust regression with $M$-estimates (and Huber's method) was used instead of OLS. Summary statistics for both the OLS and robust regression are given, respectively, in Table \ref{tab:OLS_table} and \ref{tab:Robust_table} in Appendix \ref{chapt:appendix_B}. 

%Some input variables were not significant as indicated by their $p$-values. (These variables were the NO$_2$ data from Norr Malma, precipitation, and temperature.) Removing them lowered the condition number to below 100 and improved the model, as indicated by the $R^2$-value and the RMSE (when making new predictions on the test data). Still, with many input variables, an $L_1$-regularized version of robust regression was tested \cite{Huber2009}. This model shrunk the coefficients somewhat, but only brought minimal improvements in terms of RMSE (less than one tenth of a percent), which is why the non-regularized robust regression model was deemed sufficient as a base model for comparison. The final model had the form 

%With many variables in the model, a regularized approach was tested, namely ridge regression. However, \\

%Including data (also log transformed) from other stations improved the MLR model, as did including the temporal variables for day, but the temporal variables for week and year were not significant ($p$-values $>$ 0.05). Data from other stations were, similar as for the response variable, fit with the values at lag one (since future values of the predictors cannot be known). Despite having data from several nearby stations, the condition number did not indicate any strong multicollinearity. However, when also including weather parameters in the model, the condition number rose to well above 100. Due to this, 

% lagged response variables as predictors was fit with OLS. The Durbin-Watson test showed that for both PM$_{10}$ and PM$_{2.5}$, including one response variable ($y_{t-1}$ and $y_{t-2}$) was enough to eliminate any autocorrelation, but 
%With data from three stations in the Stockholm area, some collinearity was expected, and regularized MLR models were initially tried. However, 

\subsection{Deep learning models}

\subsubsection{Model fitting}
At first, a fully connected neural network (dense model) was trained with the same input data as for the MLR model (i.e., the values at lag one and two for the response variable, lag one values for the NO$_2$ data from the other three stations, as well as the meteorological and temporal variables). Subsequent deep learning models, namely an additional dense model, a simple RNN, an LSTM, and a GRU model, were all fit with data windows of different input lengths (6, 12, and 24 hours, with the next hour as the target window), however with the same set of input variables as for the MLR and the first dense model. 

It should be emphasized that these two model fitting strategies differ in that the models with rolling windows utilize multivariate time series from the immediate past, whereas the first dense and the MLR model utilize past pollution data, as well as "real-time" data of the meteorological variables (and hence the meteorological variables would need to be replaced with forecasted values should the models be used in operational mode). Having the same set of input variables (again, being NO$_2$ data from the target site as well as three adjacent sites, four meteorological variables, and the two temporal variables) for all models allows for a more direct comparison between the baseline MLR model and the deep learning models, as well as the two different model fitting strategies. 

\subsubsection{Hyperparameter tuning}
\label{sec:tuning_deep_learning}
For all deep learning models, the hyperparameters tuned were; number of hidden layers (up to five were tested), number of units in each hidden layer (32--512, with step size 32), and learning rate (sampled in the range $[1\times10^{-5},\: 1\times10^{-2}]$). For the two dense models, ReLU was used as activation function, whereas for the RNN, LSTM, and GRU models, the tanh function was used. 

To prevent overfitting, a dropout layer was added after each hidden layer for the dense models, whereas for the simple RNN, LSTM, and GRU models, recurrent dropout were used. For all models, the dropout rate was also tuned (0.0--0.5 with step size 0.1, where a dropout rate of zero is equivalent to no dropout). The simple RNN, LSTM, and GRU models all had a dense output layer, and to regularize this as well, a dropout layer was added before the dense output (where the dropout rate was tuned as described above). 
%If a model showed signs of overfitting (to check this, the learning curves during training were consulted), a drop-out layer was added after each layer, where the dropout rate was also tuned (0.0--0.5, with step size 0.05). Moreover, input windows of different lengths (3, 6, 12, and 24 h) were tested for all models as well. 

Due to the large number of hyperparameter combinations (making it infeasible to test all within a reasonable amount of time), the Bayesian optimization tuner provided by the Keras library was used \cite{omalley2019kerastuner}. This tuner uses Gaussian processes to select hyperparameters that are likely to improve the model given previous results, and it was assumed that convergence to an optimal set of hyperparameters would be found relatively quickly (maximum number of trials were set to 50). 
%Furthermore, the same input variables as for the final MLR model were used in all deep learning models as this provides a more direct comparison of the performance between the two type of models.
As a last step, the optimal number of epochs were tuned as well, and the final models were re-trained with the best set of hyperparameters (including the best epoch), and finally evaluated on the test sets. 

The best set of hyperparameters for the two dense models are given in \cref{tab:hyperparams_dense}, and for the simple RNN, LSTM, and GRU models in \cref{tab_hyperparams_RNN}. For the models that were fit with rolling windows, the length of the input windows can be viewed as a hyperparameter as well, and in \cref{tab:hyperparams_dense} and \cref{tab:hyperparams_RNN}, only the window length (indicated in parentheses) for the best performing models are given.

% Please add the following required packages to your document preamble:
% \usepackage{booktabs}
% \usepackage{graphicx}
\begin{table}[h]
\centering
\caption{The best hyperparameters for the two dense models. }
\label{tab:hyperparams_dense}
\resizebox{\textwidth}{!}{%
\begin{tabular}{@{}cccccc@{}}
\toprule
Model      & Learning rate & \begin{tabular}[c]{@{}c@{}}Units/dropout,\\ 1st layer\end{tabular} & \begin{tabular}[c]{@{}c@{}}Units/dropout,\\ 2nd layer\end{tabular} & \begin{tabular}[c]{@{}c@{}}Units/dropout,\\ 3rd layer\end{tabular} & \begin{tabular}[c]{@{}c@{}}Units/dropout,\\ 4th layer\end{tabular} \\ \midrule
Dense      & 0.00001       & 512 / 0.0                                                          & 32 / 0.0                                                           & 512 / 0.3                                                          & 32 / 0.0                                                           \\
Dense (6h) & 0.00001       & 512 / 0.0                                                          & -                                                                  & -                                                                  & -                                                                  \\ \bottomrule
\end{tabular}%
}
\end{table}

%The Keras Tuner library \cite{omalley2019kerastuner} was used to find the best set of hyperparameters for each model (except for the MLR and ARIMA models used as baseline). 
%%Hyperparameters at both the model architecture-level as well as the input data-level were included in the search space. More specifically, at the input data-level, the width of the windows were set to 3 h, 6 h, or 12 h, and the models were fit with the different versions of the input data. 
%More specifically, the following hyperparameters were tuned:
%
%\begin{itemize}
%\item Number of layers (up to five were tested)
%\item Number of units per layer (in the range [32, 512] with step size set to 32)
%\item Learning rate (sampled uniformly in the range [0.0001, 0.01])
%\item Number of epochs 
%\end{itemize} 
%
%%For some models, a dropout layer (with rates in the range [0, 0.3] and step size set to 0.05) was also tested.
%For the hyperparameter search, Bayesian optimization was used as tuner. (The Bayesian optimization tuner tries to predict which hyperparameters that are likely to improve the model given previous results \cite{omalley2019kerastuner}). The motivation for this choice is the large number of possible hyperparameter combinations, making it infeasible to test all of them within a reasonable amount of time. Instead, it was assumed that the tuner after 75 trials would find some optimal set of hyperparameters. 
%The hyperparameter search was done in total three times for every model; one search each was performed for data input windows of different sizes, namely 8 h, 16 h, and 24 h. After completing the search, the number of epochs for each model were tuned, and all models were re-trained and evaluated on the validation and test data. 

%\begin{figure}[h]
%\centering
%\makebox[\textwidth][c]{\includegraphics[width=0.9\textwidth]{../plots/time_series_plots_7}}
%\caption{Time series plots for PM$_{10}$ and PM$_{2.5}$.}
%\label{fig:time_series_plots}
%\end{figure}

%, the models were re-trained on the training set plus the validation set, and performance on the test set was recorded. Again, with three different window sizes tested, three versions per model were obstained. All predictions were made for PM$_{10}$ one hour ahead for the station at Torkel Knutssongatan (measuring urban background levels in the center of Stockholm).


\chapter{Results and Discussion}

\section{Model evaluation}
Forecasts on the original scale of a time series are easier to interpret, and therefore the transformations made before model fitting were reversed to convert back to \textmugreek g/m$^3$ before proceeding with the model evaluation. This required inverting the normalization, and also for the MLR model, using exp($\hat{y}$) for the predictions. 
%In the sections below, common regression metrics are analyzed for all models. Also the forecast errors are looked at in detail. 

\subsection{Common regression metrics}
%In this section, commonly used regression metrics for each model are given and the results are briefly discussed. Further down however (section ??), a more formal comparison of the metrics are carried out with two statistical tests.  
The RMSE, MAE, and ME for the robust MLR model and the neural network models are shown in \cref{fig:metrics_barchart}. Again, for the neural network models that were fit with rolling windows, the length of the input windows are indicated in parenthesis, and only the best performing models (in terms of lowest RMSE) are included here. 

As can be inferred from \cref{fig:metrics_barchart}, the neural network models all had lower RMSE than the baseline MLR model, with the LSTM model having the lowest value (3.10), closely followed by the dense model (3.18) that was not fit with rolling windows (see \cref{sec:model_fitting}). Looking at the MAE, also the LSTM model had the lowest value (1.89), again followed by the dense model (1.93), however, for the three remaining neural network models, the MAE were higher than for the baseline MLR model. Though both the RMSE and the MAE measure variability in $e_t(1)$, squaring the errors before averaging (as is done with the MSE) will weight the errors differently than when taking the absolute value. More specifically, the MSE/RMSE penalize large errors more than does the MAE \cite{reg_metrics2018}.
%and the three neural network models having comparatively high MAE's (2.05, 2.09, and 2.06), but still lower RMSE's, \textcolor{red}{most likely generated better forecasts during episodes with frequent pollution peaks than did the MLR model, even though the MLR model had a lower MAE (1.99)}. 
Given the research question of this thesis, where the emphasis is on pollution peaks (where larger errors are expected), the MSE/RMSE would be the more logical measure to focus on here. 

Finally, looking at the ME in \cref{fig:metrics_barchart}, the recurrent neural network models all had small ME's, indicating considerably less bias in the forecasts than for the two dense models and the MLR model (which had the largest ME of 0.48). Not only the magnitude, but also the sign of the ME is important, as a positive ME indicates an underestimation bias, whereas a negative ME indicate bias in the opposite direction. Consequently, the robust MLR model clearly gave predictions that were too low, whereas the opposite is true for the two dense models. 

%Clearly, the recurrent models appeared to be superior in this regard. In \cref{sec:inference} further down, inference for the regression metrics are given. 

\begin{figure}[h] 
\begin{center}
\makebox[\textwidth][c]{\includegraphics[width=1\textwidth]{../plots/metrics_barchart}}
%\includegraphics[scale=1]{../plots/}
\caption{Performance metrics for all models.}
\label{fig:metrics_barchart}
\end{center}
\end{figure}

\subsection{Examining the forecast errors}
%As described in \cref{sec:deep_learning}, the RNN, LSTM, and GRU models are all recurrent neural network models. Out of these, the LSTM model showed superior performance over two common evaluation metrics (\cref{tab:performance_metrics}). Similarly, for the two dense models, the one fit with the same input data as for the MLR model had better performance compared to the one fit with rolling windows. Therefore, in the sections below where forecast errors are examined closer, the RNN and GRU models as well as the dense model fit with rolling windows are not analyzed further. Hence, the discussion will be limited to the robust MLR model, the first one of the two dense models, and the LSTM model. 

\subsubsection{Residual autocorrelations}

The Box-Pierce test was used with all models to see if the distribution of the sample ACF's for the forecast errors were approximately normal. As described in \cref{sec:auto_errors}, this is equivalent to testing whether the forecast errors are white noise. Using the 50 first autocorrelations, the Box-Pierce statistic was very large (well above 100) for all models, and the corresponding $p$-values $\ll0.001$. The results from the Box-Pierce tests are summarized in \cref{tab:boxpierce} in Appendix \ref{chapt:appendix_C}. QQ-plots and histogram plots of the forecast errors for all models are also shown in \cref{fig:qq-plot-errors} and \cref{fig:histogram_errors}, respectively, in Appendix \ref{chapt:appendix_C}. The errors were not normally distributed for any of the models, as is indicated by the shape of the histograms in \cref{fig:histogram_errors}, but also by the departures from the diagonal line in \cref{fig:qq-plot-errors}. Consequently, this suggests that the forecast errors were not Gaussian white noise. Instead, the errors appeared to be strongly correlated and non-random for all models. The structure of the errors are discussed next.

%can be seen from \cref{fig:histogram_errors} and also

\subsubsection{Structure of the forecast errors}
A more clear understanding of the structure of the forecast errors can be given by looking at the scatter plots of observed vs.\ predicted values for each model, shown in \cref{fig:correlations}. The correlation between observed and predicted values are also indicated with the Pearson correlation coefficient ($\rho$) for each model. If the forecasts were unbiased, the data points would be evenly distributed around the diagonal lines. However, this was not the case for any of the models, as a noticeable tendency to underestimate NO$_2$ levels in the higher ranges ($ \geq 40$ \textmugreek g/m$^3$) can be seen.  

% and especially in the higher NO$_2$ ranges, a noticeable tendency to underestimate NO$_2$ values is seen.

%noticeable underestimations of NO$_2$ values can be seen for all models  
%for all models, as the NO$_2$ values increase, the forecasts tended underestimate NO.
%the forecasts had a tendency to underestimate NO$_2$ values, and this becomes more noticeable as the NO$_2$ values increase. 

The MLR model had the lowest correlation coefficient ($\rho = 0.916$), and also the strongest tendency to underestimate NO$_2$ values, as indicated by the many data points located below the diagonal line in \cref{fig:correlations}a. Contrary to the neural network models (\cref{fig:correlations}b-f), the MLR model also made a few predictions that were strongly overestimated. 

For all neural network models, the correlation coefficient was higher than for the MLR model (and again the LSTM model had the best performance for this metric, $\rho=0.934$). Furthermore, the predictions were less biased, both in the lower ranges of NO$_2$ levels, but especially in the higher ranges when compared to the MLR model (which showed clear bias in the predictions already for observations exceeding 25 \textmugreek g/m$^3$). However, as pointed out above, all neural network models had the same problem of generating too low predictions at higher NO$_2$ levels (and this is very noticeable in \cref{fig:correlations}b-f). The ME's for the two dense models indicated a slight overestimation bias (\cref{fig:metrics_barchart}). However, from \cref{fig:correlations}b-c, it is clear that this bias only pertained to predictions in the lower ranges, and not the higher ranges where the opposite is true.
 
%in the lower ranges of NO$_2$ values compared to the MLR model, which shows strong bias already when observations start to exceed 25 \textmugreek g/m$^3$. In the higher ranges of NO$_2$ observations (say, 50 \textmugreek g/m$^3$ and above), all models tended to make predictions that were too low, though the bias in these ranges was much less severe for the neural network models (\cref{fig:correlations}b-f). 
\begin{figure}[h]
\centering
%\makebox[\textwidth][c]{\includegraphics[width=1.1\textwidth]{../plots/qq_pred_errors_all}}
%\caption{Normal probability plots for the forecast errors.}
%\label{fig:qq-plot-errors}
%%\end{figure}
%%\begin{figure}[]
%%\bigskip
%%\centering
%\vspace*{\floatsep}% https://tex.stackexchange.com/q/26521/5764
\makebox[\textwidth][c]{\includegraphics[width=1\textwidth]{../plots/correlations_all}}
\caption{Scatterplots of observed vs.\ predicted NO$_2$ values.}
\label{fig:correlations}
\end{figure}
%\clearpage

\noindent
In \vref{fig:pred_vs_obs}, observed and predicted NO$_2$ levels during a 10 day period in December 2021 are shown for the robust MLR model (\cref{fig:pred_vs_obs}a), and the LSTM model (\cref{fig:pred_vs_obs}b). Only the best performing model is compared to the baseline MLR model in \cref{fig:pred_vs_obs}, since the forecast errors were similar in character for all models. From these plots, the structure of the forecast errors also become apparent (and merely corroborate the findings related to \cref{fig:correlations}). For example, it can be seen that both models had difficulties predicting the magnitude, as well as duration of, the large NO$_2$ peak occurring around 7--8 December, though the forecasts of the LSTM model are still better. More accurate forecasts for the LSTM model are also seen during the smaller NO$_2$ peaks (e.g.\ the peaks occuring around 3--4 and 6--7 December). At low NO$_2$ levels however, predictions for both models follow the observed NO$_2$ values closely. 

%When looking at the predictions for lower NO$_2$ levels, it is not clear whether the LSTM model . 

\begin{figure}[h] 
\begin{center}
\makebox[\textwidth][c]{\includegraphics[width=1.02\textwidth]{../plots/predictions}}
%\includegraphics[scale=1]{../plots/predictions}
\caption{Observed and predicted NO$_2$ levels during 10 days in December.}
\label{fig:pred_vs_obs}
\end{center}
\end{figure}

\section{Inference for the regression metrics}
\label{sec:inference}

Due to the fact that the forecast errors for none of the models were Gaussian white noise (and consequently not normally distributed), the Kruskal-Wallis test was used to test whether there was a significant difference between the models with two of the performance metrics, namely the MSE and the ME. The test indicated a significant difference, both for the MSE ($\chi^2(5) = 299.65$, $p\ll0.001$) and the ME ($\chi^2(5) = 593.40$, $p\ll0.001$), between the different models. 

Following this, Dunn's post-hoc test with Bonferroni adjustment was used to test pairwise comparisons of the models. Again, both measures (MSE and the ME) were tested. For the MSE, the main finding was that there were significant differences between the baseline MLR model and all neural network models (all $p$-values were $\ll0.05$). Between the two best performing models however (the LSTM and the dense model), no significant difference was indicated ($p> $ 0.05). For the ME, there were also significant differences between the MLR model and all neural network models (all $p$-values $\ll0.001$). Moreover, significant differences between all three recurrent neural network models and the two dense models were also indicated ($p$-values $\ll0.001$). (Given the metrics from \cref{fig:metrics_barchart}, this was expected). All $p$-values for the pairwise comparisons from the Dunn's test for the MSE and the ME are summarized in \cref{tab:Dunn_MSE} and \cref{tab:Dunn_ME}, respectively, in Appendix \ref{chapt:appendix_C}. 

\section{Summary and discussion}
Common regression metrics indicated that all neural network models generated better forecasts than the baseline MLR model. Moreover, the results were statistically significant, as indicated by the Kruskal-Wallis and the Dunn's test. However, the improvements were not dramatic, as the best performing neural network model generated forecasts with roughly 12\% lower RMSE and about 5\% lower MAE compared to the baseline model. (As pointed out above, though not insignificant, less emphasis is put on the MAE here.) 

A more noticeable improvement is seen for the ME, for which all RNN models showed superior performance compared to the baseline, as well as the two dense models. Clearly, the RNN models managed to utilize temporal dependencies in the data in a way so to reduce the bias. However, any further advantages for the RNN models are unclear, as there was no significant difference between the best performing dense model and the LSTM model when looking at the RMSE. Here, the nature of the time series being modeled is relevant. For example, if temporal dependencies only are important up to a few past time steps (or just one time step) when the next value is to be predicted, a simpler model not explicitly taking the history of the time series into account will produce successful predictions as well. An indication of the importance of previous time steps for the response variable can be given by looking at the magnitude of the coefficients (or the $z$-statistic) for the robust MLR model (\cref{tab:Robust_table} in Appendix \ref{chapt:appendix_B}). It can be inferred that the value at lag one dominated in importance, whereas the value at lag two, though still significant, mattered much less. Values of the response variable at lags further back were not utilized in the  MLR model, but it can be reasonably assumed that the importance decrease at every time step going back. Another interesting finding is that for the LSTM model, using the 12h input window gave better results than with the 24h input window. (For the GRU model, the 24h input window gave better predictions, the improvements over the 12h input window were only subtle however). This also suggests that longer-term temporal dependencies were of less importance for predicting NO$_2$ levels the next hour. 

Despite the performance improvements seen for the neural network models (especially the LSTM model), all models failed to successfully capture the structure in the observations, as there was strong autocorrelation in the forecast errors. Judging from \cref{fig:correlations} and \cref{fig:pred_vs_obs}, the autocorrelation in the errors are due to the poor predictions made at high NO$_2$ levels, and especially jerky leaps like the large peak seen around 7--8 of December in \cref{fig:pred_vs_obs} are not captured well. Also, the two dense models tended to generate overestimated predictions at lower NO$_2$ levels, as indicated by the ME's (though this is not clear from \cref{fig:correlations}b-c).

An additional aspect that should be discussed is that of the two model fitting strategies. As explained in \cref{sec:model_fitting_hp_tuning}, the MLR model and the first dense model were not fit with lagged meteorological variables, and in operational mode these values would need to be replaced with forecasts. Consequently, the performance metrics reported here are under the assumption of meteorological forecasts that are exact. Clearly, this is an unreasonable assumption, and the theoretical performance of these models are expected to be too optimistic. On the other hand, the models fit with rolling windows only utilize past pollution and weather data, and therefore do not suffer from this added uncertainty. Hence, the theoretical performance for these models should better reflect the expected performance in operational mode. 

% as pointed out above, in this work, more importance is given to the RMSE measure, and it can also be noted that the MSE error is the quantity being minimized during training.) 

%However, more striking is the difference in ME between the baseline MLR and the three recurrent neural network models, where for the 

%
%The main findings are briefly described here, and in \cref{tab:Dunn_MSE} and \cref{tab:Dunn_ME} in Appendix \ref{chapt:appendix_C}, the results from both tests are summarized. 

%For the MSE, there was a significant difference between the robust MLR model and all the neural network models. Looking specifically at the neural network models, there was no significant difference between the two best performing models (the LSTM and the dense model), nor was it any significant difference between the remaining three models (the RNN, GRU, and dense fit with rolling windows). 

%however, for both of these models, there was a significant difference between the three remaining 


%there was also a significant difference between the two best performing models (the LSTM and the dense model) and the remaining three (the simple RNN, the GRU, and the dense fit with rolling windows). 

%\cref{tab:Dunn_MSE} and \ref{tab:Dunn_ME} in Appendix \ref{chapt:appendix_C} summarizes the results. 



%The largest spread in the data points can be seen for the MLR model (\cref{fig:correlations}a), which also had the lowest correlation coefficient ($\rho = 0.916$). 
%
%
%as well as the highest ME (\cref{tab:performance_metrics}). Contrary to the neural network models, the MLR model also made some predictions that were strongly over-estimated. 
%
%All neural netwsork showed stronger correlation between observed and predicted values than , and also for this metric, the LSTM model showed the best performance ($\rho = 0.934$).

%and the Box-Pierce statistic together with corresponding $p$-values for the robust MLR model as well as the dense and LSTM models are given in \ref{tab:box-pierce}. 


%For all deep learning architectures, the 12 h input window gave better predictions than with the 24 h input window, and in \cref{tab:metrics}, common performance metrics for the best performing deep learning models are summarized together with the corresponding performance for the MLR model.

%From \cref{tab:metrics}, it can be inferred that the dense model had the best performance in terms of MSE/RMSE, closely followed by the LSTM model. The RNN model, however, only had slightly better performance than the (baseline) MLR model. Interestingly, the MAPE follows an entirely different pattern, where the MLR model had the lowest value (20\%), and the LSTM model the highest (29.6\%). This indicates a stronger bias in the predictions by the deep learning models compared to the MLR model. 


%In \cref{fig:correlations}, where observed vs.\ predicted NO$_2$ levels are shown for each model together with the corresponding correlation coefficient, the bias in the predictions for the deep learning models can be seen. For example, looking at plots (b)--(d), there is a consistent pattern of too high predictions (indicated by the unequal distribution of observations around the red line). For the MLR model (\cref{fig:correlations}a), though less biased, the bias goes in the opposite direction, with a tendency to make too low predictions. The correlation coefficients followed the same pattern as MSE/RMSE, with the dense model having the highest $\rho$. 

%In \cref{fig:predictions}, predicted and observed NO$_2$ levels for all models during a 9 day window (1st to 9th of Dec) from the test set are shown. Comparing the MLR model with the deep learning models in \cref{fig:predictions}, it is clear that the MLR model performed worse during pollution peaks (e.g.\ the peak observed around 7--8 of Dec was not predicted very accurately). The model best predicting peaks during this time window appears to be the RNN model, but again the RNN model is clearly biased and also had the lowest $\rho$ among the deep learning models. 

% sine and cosine for hours important for hourly predictions... but day and year unimportant


\chapter{Conclusions}
% MLR model simple, but lots of data prep

% deep learning models not as much data prep (and checking model assumptions), but longer to train

% autoregressive version could be tried, but would require predicting/forecasting NO2 values att ALL stations

% NO2 also modeled well with a simple linear techniqe sometimes a simpler model can do the job....
% variance increase as levels increase... not good for MLR? make stattionary with ARIMA models a better idea?
% confodence indervals
% additional several steps forecasts..

In this work, several different deep neural network architectures have been explored and compared with a multiple linear regression model for predicting hourly urban background levels of NO$_2$ using time series data. Generally, the multiple linear regression model was straightforward to implement compared to the deep neural network models, though additional data preparation steps were necessary to ensure that the model assumptions were not violated. The deep neural network models required less data preparation, but took much longer to train and tune. 

Across several evaluation metrics, the deep neural network models performed better than the multiple linear regression model, and in particular, a recurrent neural network model (LSTM) consistently had superior performance. The theoretical performance for the best model should also reflect the expected performance in operational mode. Viewing the results in relation to the research question of the thesis (\cref{sec:research question}), sudden NO$_2$ peaks were poorly predicted however, also by the best model, and for all models, forecasts at high NO$_2$ levels were unsatisfactory. Moreover, none of the models had the desired structure of the forecasts errors, which warrants further model refinements. 

Generally when forecasting air pollution, it is common to focus on daily mean levels rather than hourly means \cite{atmos7020015, Shams2021}. This makes comparisons to other work difficult, but the results herein are similar to that of e.g.\ Rahimi, 2017 \cite{Rahimi2017}, Goulier et al., 2020\cite{Goulier2020} and Arsov et al., 2021 \cite{Arsov2021}. Also SLB-analys' forecasts for the Stockholm metropolitan area are given as risk indexes based on daily means of pollutants \cite{slbanalys}. Hourly forecasts though, has the advantage over daily forecasts in that they can indicate during what hours of the day pollution levels could reach hazardous levels, and this knowledge could in some cases be critical. The forecast horizons here are very short-term (one hour), which admittedly limits usability. However, all models in this work could quite easily be adapted for forecast horizons of arbitrary length by simply utilizing the predictions generated. Also, the models could of course be used with other type of air pollutants (PM$_{2.5}$, PM$_{10}$, O$_3$, etc.) and in which case, if all are forecast simultaneously, would allow for prediction of an hourly air quality index \cite{VanLoon2010}. Such detailed forecasts could be of great societal value; they could e.g.\ be used by public health authorities and policy makers. If the forecasts also were made available to the public, they could help with early warning systems aimed at sensitive groups.

Utilizing time series data from multiple air pollution sensors is a challenging task, and this work points to an advantage for the more complex neural networks over the simpler linear model for this type of problem. 
%and many other modeling and forecasting techniques than the ones herein could also be used for this type of problem. For example, as pointed out in \cref{sec:DL models}, convolutional neural networks are often employed to model spatial features in data, and these networks can be combined with other type of networks to form powerful hybrid-network architectures. There are also convolutional networks tailored to time series \cite{Liao2020, He2019}. 
%As the machine learning field advances, and the availability of massive datasets from multi-sensor air monitoring systems and other relevant data sources (like weather, traffic, and satellite data) increases, 
%
%forecasts are expected to improve, and current limitations in forecasting. 
Further advances in the field of machine learning (specifically deep learning), together with increased availability of massive datasets from multi-sensor air monitoring systems and other relevant data sources (like weather, traffic, and satellite data), offers the potential to cross the conventional limits of forecasting in the near future. 

% the implications...
% vulnerable groups can plan their day to day activiies better based on hourly forecasts..

%and as data from multi-sensor air monitoring systems together with data from other relevant sources (like weather, traffic, satellite data) become increasingly, producing massive dataset, the current limitations in forecasting are.

%the challenges of utilizing the ever increasing amount of data (weather, traffic, )

%As air sensors are becoming cheaper and more available, an ever increasing amount of air pollution as well as weather data is produced.  

%together with an increase in the availability of massive datasets from multi-sensor air monitoring systems and other relevant sources (like traffic and satellite data), conventional limits 

%due to an increased availability of data from multi-sensor air monitoring systems and other 

%air pollution sensors, the accuracy of forecasts

%Also, computationally less expensive methods such as Gaussian processes can often deal with the challenges that spatio-temporal data presents \cite{smlbook, FaganeliPucer2018}.

%For example, Gaussian processes are frequently used to deal with the challenges of spatio-temporal data \cite{smlbook, FaganeliPucer2018}. Other neural network models, such as convolutional neural networks, are also frequently employed for modeling spatial features in data, and various deep learning architectures where for example the output of a convolutional neural network is subsequently fed into a recurrent neural network have been tried \cite{Liao2020}. Also, utilizing additional information sources such as motor traffic data (travel speeds, traffic flow and intensity, etc.) and satellite image data could further help improve air quality forecasts \cite{atmos7020015, Liao2020}. 

%Nevertheless, in this work, the deep learning models showed promising results were 

%(i.e., by plugging them back into the models, so for example the forecast for time step $t+2$ would be partially based on the forecast made for time step $t+1$). 
 
% EU goial of AI - this work aligns perfectly with that

% machine learningg has the potential to improve human life and prosper....



% references section
\bibliography{references}{}
\bibliographystyle{ieeetr}
%\bibliographystyle{agsm}

%\appendix
%\chapter{Monitoring Stations}
%\label{chapt:appendix_A}
% STATIONS TABLE
% ##############
Information about the monitoring stations from which data was used is summarized in \vref{tab:stations}. In \vref{fig:time_series_plots_all}, time series plots of PM$_{10}$ and PM$_{2.5}$ at each station are shown. 
\begin{table}[h]
\centering
\caption{Monitoring stations.}
\label{tab:stations}
\resizebox{\textwidth}{!}{%
\begin{tabular}{@{}llllll@{}}
\toprule
Station                         & Station code & Longitude & Latitude  & Type of monitoring                                                   & Parameters                                                                                             \\ \midrule
Norrtälje, Norr Malma           & 18643        & 18.631313 & 59.832382 & \begin{tabular}[c]{@{}l@{}}Rural-Regional \\ Background\end{tabular} & PM$_{10}$, PM$_{2.5}$                                                                                  \\ \midrule
Stockholm, Hornsgatan 108       & 8780         & 18.04866  & 59.317223 & Urban Traffic                                                        & PM$_{10}$, PM$_{2.5}$                                                                                  \\ \midrule
Stockholm, Torkel Knutssonsgatan & 8781         & 18.057808 & 59.316006 & Urban background                                                     & \begin{tabular}[c]{@{}l@{}}PM$_{10}$, PM$_{2.5}$, NO$_2$, \\ meteorological \\ parameters\end{tabular} \\ \bottomrule
\end{tabular}%
}
\end{table}

%% ##############
%\begin{figure}[h]
%\centering
%\makebox[\textwidth][c]{\includegraphics[width=1\textwidth]{../plots/time_series_plots_all}}
%%\includegraphics[width=\textwidth]{../plots/time_series_plots}
%\caption{Time series plots for PM$_{10}$ and PM$_{2.5}$ at all stations.}
%\label{fig:time_series_plots_all}
%\end{figure}

\pagebreak
\chapter*{Appendices}
\counterwithin{figure}{section}
\counterwithin{table}{section}
%% provide three setup instructions:
\addcontentsline{toc}{chapter}{Appendices} % write to the toc file
\setcounter{section}{0}
\renewcommand\thesection{\Alph{section}}

\section{Monitoring stations}
\label{chapt:appendix_A}
% STATIONS TABLE
% ##############
Information about the monitoring stations from which data was used is summarized in \vref{tab:stations}. In \vref{fig:time_series_plots_all}, time series plots of PM$_{10}$ and PM$_{2.5}$ at each station are shown. 
\begin{table}[h]
\centering
\caption{Monitoring stations.}
\label{tab:stations}
\resizebox{\textwidth}{!}{%
\begin{tabular}{@{}llllll@{}}
\toprule
Station                         & Station code & Longitude & Latitude  & Type of monitoring                                                   & Parameters                                                                                             \\ \midrule
Norrtälje, Norr Malma           & 18643        & 18.631313 & 59.832382 & \begin{tabular}[c]{@{}l@{}}Rural-Regional \\ Background\end{tabular} & PM$_{10}$, PM$_{2.5}$                                                                                  \\ \midrule
Stockholm, Hornsgatan 108       & 8780         & 18.04866  & 59.317223 & Urban Traffic                                                        & PM$_{10}$, PM$_{2.5}$                                                                                  \\ \midrule
Stockholm, Torkel Knutssonsgatan & 8781         & 18.057808 & 59.316006 & Urban background                                                     & \begin{tabular}[c]{@{}l@{}}PM$_{10}$, PM$_{2.5}$, NO$_2$, \\ meteorological \\ parameters\end{tabular} \\ \bottomrule
\end{tabular}%
}
\end{table}

%% ##############
%\begin{figure}[h]
%\centering
%\makebox[\textwidth][c]{\includegraphics[width=1\textwidth]{../plots/time_series_plots_all}}
%%\includegraphics[width=\textwidth]{../plots/time_series_plots}
%\caption{Time series plots for PM$_{10}$ and PM$_{2.5}$ at all stations.}
%\label{fig:time_series_plots_all}
%\end{figure}

\pagebreak
\section{Model diagnostics and summary statistics for the multiple linear regression models}
Diagnostics plots are shown in \ref{fig:residuals_MLR_PM10} and \ref{fig:residuals_MLR_PM2.5} below and models statistics are shown in tables... 

%\begin{figure}[h]
%\centering
%\makebox[\textwidth][c]{\includegraphics[width=0.95\textwidth]{../plots/Residual_plots_MLR_PM10}}
%\caption{Residual plots for the PM$_{10}$ MLR model.}
%\label{fig:residuals_MLR_PM10}
%\end{figure}
%
%\begin{figure}[h]
%\centering
%\makebox[\textwidth][c]{\includegraphics[width=0.95\textwidth]{../plots/Residual_plots_MLR_PM2.5}}
%\caption{Residual plots for the PM$_{2.5}$ MLR model.}
%\label{fig:residuals_MLR_PM2.5}
%\end{figure}

% PM10
\begin{landscape}
\begin{table}[h]
\begin{center}
\begin{tabular}{lclc}
\toprule
\textbf{Dep. Variable:}                           & PM$_{10}$, Torkel Knutssonsgatan & \textbf{  R-squared:         } &     0.755   \\
\textbf{Model:}                                   &                 OLS                 & \textbf{  Adj. R-squared:    } &     0.755   \\
\textbf{Method:}                                  &            Least Squares            & \textbf{  F-statistic:       } & 2.026e+04   \\
\textbf{Date:}                                    &           Thu, 11 Aug 2022          & \textbf{  Prob (F-statistic):} &     0.00    \\
\textbf{Time:}                                    &               23:46:52              & \textbf{  Log-Likelihood:    } &   -10505.   \\
\textbf{No. Observations:}                        &                 26328               & \textbf{  AIC:               } & 2.102e+04   \\
\textbf{Df Residuals:}                            &                 26323               & \textbf{  BIC:               } & 2.106e+04   \\
\textbf{Df Model:}                                &                     4               & \textbf{                     } &             \\
\textbf{Covariance Type:}                         &              nonrobust              & \textbf{                     } &             \\
\bottomrule
\end{tabular}
\begin{tabular}{lcccccc}
                                                  & \textbf{coef} & \textbf{std err} & \textbf{t} & \textbf{P$> |$t$|$} & \textbf{[0.025} & \textbf{0.975]}  \\
\midrule
\textbf{intercept}                                &       0.0656  &        0.009     &     6.974  &         0.000        &        0.047    &        0.084     \\
\textbf{PM$_{10}$, Torkel Knutssonsgatan lag1} &       0.6335  &        0.007     &    96.861  &         0.000        &        0.621    &        0.646     \\
\textbf{PM$_{10}$, Torkel Knutssonsgatan lag2} &       0.1072  &        0.006     &    17.494  &         0.000        &        0.095    &        0.119     \\
\textbf{PM$_{10}$, Hornsgatan lag1}            &       0.1234  &        0.004     &    30.778  &         0.000        &        0.116    &        0.131     \\
\textbf{PM$_{10}$, Norr Malma lag1}            &       0.0808  &        0.004     &    19.644  &         0.000        &        0.073    &        0.089     \\
\bottomrule
\end{tabular}
\begin{tabular}{lclc}
\textbf{Omnibus:}       & 7362.227 & \textbf{  Durbin-Watson:     } &     1.976   \\
\textbf{Prob(Omnibus):} &   0.000  & \textbf{  Jarque-Bera (JB):  } & 122434.478  \\
\textbf{Skew:}          &  -0.906  & \textbf{  Prob(JB):          } &      0.00   \\
\textbf{Kurtosis:}      &  13.408  & \textbf{  Cond. No.          } &      21.8   \\
\bottomrule
\end{tabular}
\caption{OLS Regression Results for PM$_{10}$}
\end{center}
\end{table}
\end{landscape}

% PM2.5
\begin{landscape}
\begin{table}[h]
\begin{center}
%\resizebox{\textwidth}{!}{%
\begin{tabular}{lclc}
\toprule
\textbf{Dep. Variable:}                            & PM$_{2.5}$, Torkel Knutssonsgatan & \textbf{  R-squared:         } &     0.927   \\
\textbf{Model:}                                    &                 OLS                  & \textbf{  Adj. R-squared:    } &     0.927   \\
\textbf{Method:}                                   &            Least Squares             & \textbf{  F-statistic:       } & 8.370e+04   \\
\textbf{Date:}                                     &           Thu, 11 Aug 2022           & \textbf{  Prob (F-statistic):} &     0.00    \\
\textbf{Time:}                                     &               23:20:09               & \textbf{  Log-Likelihood:    } &    4594.4   \\
\textbf{No. Observations:}                         &                 26328                & \textbf{  AIC:               } &    -9179.   \\
\textbf{Df Residuals:}                             &                 26323                & \textbf{  BIC:               } &    -9138.   \\
\textbf{Df Model:}                                 &                     4                & \textbf{                     } &             \\
\textbf{Covariance Type:}                          &              nonrobust               & \textbf{                     } &             \\
\bottomrule
\end{tabular}
%}
%\resizebox{\textwidth}{!}{%
\begin{tabular}{lcccccc}
                                                   & \textbf{coef} & \textbf{std err} & \textbf{t} & \textbf{P$> |$t$|$} & \textbf{[0.025} & \textbf{0.975]}  \\
\midrule
\textbf{intercept}                                 &       0.0386  &        0.003     &    11.092  &         0.000        &        0.032    &        0.045     \\
\textbf{PM$_{2.5}$, Torkel Knutssonsgatan lag1} &       1.1199  &        0.007     &   167.681  &         0.000        &        1.107    &        1.133     \\
\textbf{PM$_{2.5}$, Torkel Knutssonsgatan lag2} &      -0.2116  &        0.006     &   -35.111  &         0.000        &       -0.223    &       -0.200     \\
\textbf{PM$_{2.5}$, Hornsgatan, lag1}           &       0.0340  &        0.004     &     8.867  &         0.000        &        0.026    &        0.041     \\
\textbf{PM$_{2.5}s$, Norr Malma, lag1}           &       0.0218  &        0.001     &    15.576  &         0.000        &        0.019    &        0.025     \\
\bottomrule
\end{tabular}
%}
\begin{tabular}{lclc}
\textbf{Omnibus:}       & 5704.682 & \textbf{  Durbin-Watson:     } &     1.987   \\
\textbf{Prob(Omnibus):} &   0.000  & \textbf{  Jarque-Bera (JB):  } & 161104.434  \\
\textbf{Skew:}          &  -0.393  & \textbf{  Prob(JB):          } &      0.00   \\
\textbf{Kurtosis:}      &  15.093  & \textbf{  Cond. No.          } &      21.3   \\
\bottomrule
\end{tabular}
\caption{OLS Regression Results for PM$_{2.5}$}
\end{center}
\end{table}
\end{landscape}

\section{Histogram and qq-plots of the forecast errors, Box-Pierce test results, and results from the Dunn's test}
\label{chapt:appendix_C}

The results from the Box-Pierce tests for all models are summarized in \cref{tab:boxpierce} below. 
\begin{table}[h]
\small
\centering
\caption{Results from the Box-Pierce tests.}
\label{tab:boxpierce}
\begin{tabular}{@{}lll@{}}
\toprule
Model                  & $Q_{\text{BP}}$ & $p$-value             \\ \midrule
Robust MLR model       & 1520.07         & $\ll0.001$ \\
Dense model            & 866.76          & $\ll0.001$  \\
Dense model (6h)       & 607.94          & $\ll0.001$ \\
Simple RNN model (12h) & 238.62          & $\ll0.001$  \\
LSTM model (12h)       & 315.34          & $\ll0.001$  \\
GRU model (24h)        & 506.06          & $\ll0.001$ \\ \bottomrule
\end{tabular}
\end{table}

\noindent
The $p$-values from the Dunn's post hoc test with Bonferroni adjustment for the MSE and the ME measures are given in \cref{tab:Dunn_MSE} and \cref{tab:Dunn_ME}, respectively. 

%MSE
% Please add the following required packages to your document preamble:
% \usepackage{booktabs}
% \usepackage{graphicx}
\begin{table}[h]
\centering
\caption{$p$-values for the pairwise comparisons from the Dunn's post hoc test for the MSE. }
\label{tab:Dunn_MSE}
\resizebox{\textwidth}{!}{%
\begin{tabular}{@{}lllllll@{}}
\toprule
Model            & Robust MLR & Dense      & Dense (6h) & Simple RNN (12h) & LSTM (12h) & GRU (24h)  \\ \midrule
Robust MLR       & 1.0        & 0.00752    & $\ll0.001$ & $\ll0.001$       & 0.00209    & $\ll0.001$ \\
Dense            & 0.00752    & 1.0        & $\ll0.001$ & $\ll0.001$       & 1.0        & $\ll0.001$ \\
Dense (6h)       & $\ll0.001$ & $\ll0.001$ & 1.0        & 1.0              & $\ll0.001$ & 1.0        \\
Simple RNN (12h) & $\ll0.001$ & $\ll0.001$ & 1.0        & 1.0              & $\ll0.001$ & 1.0        \\
LSTM (12h)       & 0.00209    & 1.0        & $\ll0.001$ & $\ll0.001$       & 1.0        & $\ll0.001$ \\
GRU (24h)        & $\ll0.001$ & $\ll0.001$ & 1.0        & 1.0              & $\ll0.001$ & 1.0        \\ \bottomrule
\end{tabular}%
}
\end{table}

%ME
\begin{table}[h]
\centering
\caption{$p$-values for the pairwise comparisons from the Dunn's post hoc test for the ME. }
\label{tab:Dunn_ME}
\resizebox{\textwidth}{!}{%
\begin{tabular}{lllllll}
\hline
Model            & Robust MLR & Dense      & Dense (6h) & Simple RNN (12h) & LSTM (12h) & GRU (24h)  \\ \hline
Robust MLR       & 1.0        & $\ll0.001$ & $\ll0.001$ & $\ll0.001$       & $\ll0.001$ & $\ll0.001$ \\
Dense            & $\ll0.001$ & 1.0        & 0.13083    & $\ll0.001$       & $\ll0.001$ & $\ll0.001$ \\
Dense (6h)       & $\ll0.001$ & 0.13083    & 1.0        & $\ll0.001$       & $\ll0.001$ & $\ll0.001$ \\
Simple RNN (12h) & $\ll0.001$ & $\ll0.001$ & $\ll0.001$ & 1.0              & 1.0        & 0.57372    \\
LSTM (12h)       & $\ll0.001$ & $\ll0.001$ & $\ll0.001$ & 1.0              & 1.0        & 0.83808    \\
GRU (24h)        & $\ll0.001$ & $\ll0.001$ & $\ll0.001$ & $\ll0.001$       & 0.83808    & 1.0        \\ \hline
\end{tabular}%
}
\end{table}

Quantile-quantile plots and histogram plots of the forecast errors are shown, respectively, in \cref{fig:qq-plot-errors} and \cref{fig:histogram_errors} below. It can be inferred from these plots that none of the models generated forecasts with the desired structure of the errors (i.e., Gaussian white noise).

% forecast errors normal probability plots
\begin{figure}[h]
\centering
\makebox[\textwidth][c]{\includegraphics[width=1.025\textwidth]{../plots/qq_pred_errors_all}}
\caption{Normal probability plots for the forecast errors.}
\label{fig:qq-plot-errors}
%\end{figure}
%\begin{figure}[]
%\bigskip
%\centering
\vspace*{\floatsep}% https://tex.stackexchange.com/q/26521/5764
\makebox[\textwidth][c]{\includegraphics[width=1.025\textwidth]{../plots/pred_errors_hist_all}}
\caption{Histogram for the forecast errors.}
\label{fig:histogram_errors}
\end{figure}

\end{document}