%\documentclass[12pt, twoside]{report}
\documentclass[12pt]{report}

\usepackage[utf8]{inputenc}
\usepackage{graphicx}
\graphicspath{ {figures/} }
\usepackage[font=small,labelfont=bf]{caption}
%\usepackage{caption}
 \usepackage{lscape}

%\usepackage[a4paper,width=150mm,top=25mm,bottom=25mm,bindingoffset=6mm]{geometry}
\usepackage[a4paper, width=155mm, top=24mm, bottom=24mm]{geometry}

\usepackage{amsmath}
\usepackage{mathtools}
\numberwithin{equation}{section}

\usepackage{booktabs} % table package
%\usepackage{tabularx} % table package for width
\usepackage{lscape}
\usepackage{fancyhdr}
\usepackage[nottoc,numbib]{tocbibind} % references in table of contents
\usepackage[euler]{textgreek}
\usepackage{upgreek}
\usepackage{varioref}
\usepackage{hyperref}
\usepackage[capitalise]{cleveref}
\usepackage[hang,flushmargin]{footmisc} % no indent in footnote

\usepackage{siunitx}
%\usepackage{amsmath}
\usepackage{float} % use H as argument to force figure placement
\usepackage[dvipsnames]{xcolor} % color in text
\usepackage{bm}

\usepackage{chngcntr}
\usepackage{titlesec} % allowing subsubsections




%\pagestyle{fancy}
%\fancyhead{}
%\fancyhead[LE, RO]{Research project}
%\fancyfoot{}
%\fancyfoot[LE, RO]{\thepage}
%\renewcommand{\headrulewidth}{0.4pt}
%\renewcommand{\footrulewidth}{0.4pt}
%\showthe\textwidth

\usepackage{cite}

% keeping report format without including chapters
% https://tex.stackexchange.com/questions/62516/how-to-suppress-chapter-in-chapter-while-keeping-numbering
\makeatletter
\def\@makechapterhead#1{%
  \vspace*{50\p@}%
  {\parindent \z@ \raggedright \normalfont
    \interlinepenalty\@M
    \Huge\bfseries  \thechapter.\quad #1\par\nobreak
    \vskip 40\p@
  }}
\makeatother


\begin{document}

% TITLEPAGE

\begin{titlepage}
    \begin{center}
        \vspace*{2cm}
            
        \Huge
%        \textbf{Forecasting Air Pollution With Machine Learning}
        \textbf{A Machine Learning Approach to Air Pollution Forecasts}
            
        \vspace{0.25cm}
        
        \LARGE
            
        \vspace{1cm}
            
        \textbf{Simon Carlén}
            
        %\vfill
        \vspace{250pt plus 1pt minus 1pt}
            
            
        %\includegraphics[width=0.3\textwidth]{university}
         
        	\vspace{.75cm}
	\large

	\begin{minipage}{0.65\textwidth}
	Degree project, 15 credits \\\indent
	Computer and Systems Sciences \\\indent
	Degree project at the master level \\\indent
	Spring term 2022 \\\indent 
	Supervisor: Sindri Magnússon \\\indent
	Co-supervisor: Ali Beikmohammadi
	%Swedish title: Luftföroreningsprognoser med djupinlärning
	\end{minipage}
	\begin{minipage}{0.32\textwidth}
	\begin{center}
    		\includegraphics[width=0.8
    		\textwidth]{university}
	\end{center}
	\end{minipage}

    \end{center}
\end{titlepage}

% ABSTRACT
\chapter*{Abstract}
\thispagestyle{empty}
The abstract and the keywords should not exceed the limit of this page
%\newline\newline
%\emph{Keywords:} Keywords should be written in order of relevance

% SYNOPSIS
\chapter*{Synopsis}
\thispagestyle{empty}

\paragraph{Background}
To mitigate the harmful effects of air pollution, air quality is regularly monitored. Such monitoring produces massive amounts of data, and this enables the development of statistical and machine learning techniques for modeling and forecasting of air pollution.
\paragraph{Problem}
Air pollution is a complex phenomenon depending on many factors, and the data from air monitoring has both temporal and spatial dependencies. This makes modeling and forecasting a challenge, and the research problem here is: \textit{To capture and model the complex dynamics of air pollution with machine learning methods, with an emphasis on deep learning}.

\paragraph{Research Question}
The research question for the thesis is: \textit{How can machine learning, in particular deep learning, be used to forecast air pollution levels and pollution peaks?} The research question emphasizes pollution peaks as these are much harder to accurately predict than pollution levels in lower ranges.  %Also, sudden jerky leaps, where pollution levels quickly rise tend to give strongly underestimated predictions, and this can have undesired consequences
%
%All type of forecasts are always imprecise, and generally for air pollution forecasts, the largest errors are seen during episodes with high pollution levels. Also, sudden jerky leaps, where pollution levels quickly rise tend to give strongly underestimated predictions, and this can have undesired consequences. The research question in this work are therefore as follows: 

\paragraph{Method}
Data was downloaded from SMHI's centralized database for air measurements, carefully examined, and thereafter preprocessed before a linear regression and several deep learning models were fit. All models were trained with historical data and later evaluated on the most recent (test) data. An important aspect of model evaluation was a close examination of the forecast errors. The Python programming language was used together with libraries for scientific programming and data science/machine learning.

\paragraph{Result}
All deep learning models outperformed the linear regression model. However, for all models, the structure of the forecast errors were not as desired, and this warrants further model refinements. The structure of the errors were due to the inability of the models to capture pollution peaks well. Though the deep learning models showed promising potential, in light of the research question, the results were unsatisfactory. 

\paragraph{Discussion}
The forecast horizons in this work (one hour) are very short-term, which limits usability. To this end however, adaptations to extend the forecast horizon are possible. Further adaptations to include more than just one air pollutant are also possible; this would result in very comprehensive forecasts that could be of great value to public health authorities and policy makers as they could permit early interventions in order to protect public health, and vulnerable groups in particular.

% ACKNOWLEDGEMENTS
\chapter*{Acknowledgements}
\thispagestyle{empty}

\tableofcontents
\thispagestyle{plain}
\thispagestyle{plain}
\setcounter{page}{1}
\pagenumbering{roman}

\listoffigures
%\setcounter{page}{1}
%\pagenumbering{roman}
\thispagestyle{plain}

\listoftables
\thispagestyle{plain}

\chapter*{List of Abbreviations}
\begin{tabular}{l@{$\dots\dots$}p{12cm}}
%Convolutional neural network\dotfill  & CNN \\
Environment and Health Administration \dotfill & EHA \\
Fully connected neural network\dotfill  & fcNN \\
Gated recurrent unit \dotfill & GRU \\
Long short-term memory \dotfill & LSTM \\
Mean absolute error\dotfill & MAE \\
Mean error \dotfill & ME \\
Mean squared error\dotfill & MSE \\
Multiple linear regression\dotfill & MLR \\
Ordinary least squares \dotfill & OLS \\
Recurrent neural network\dotfill & RNN \\
Root mean squared error\dotfill & RMSE\\
Particulate matter \dotfill & PM \\
Swedish Meteorological and Hydrological Institute\dotfill & SMHI \\
Stockholms Luft- och Bulleranalys \dotfill & SLB-analys \\
Volatile organic compounds \dotfill & VOCs \\
World Health Organization \dotfill & WHO \\
\end{tabular}

\addcontentsline{toc}{chapter}{List of Abbreviations}

\addtocontents{toc}{\bigskip}


\chapter{Introduction}
\setcounter{page}{1}
\pagenumbering{arabic}

\section{Background}

Outdoor air pollution is a major global environmental issue, linked to several serious health conditions, and causing millions of premature deaths every year \cite{who2016}. Some principal air pollutants damaging to health include gaseous substances such as nitrogen oxides (NO$_x$), ground-level ozone (O$_3$), sulphur dioxide (SO$_2$), and carbon monoxide (CO), but also atmospheric aerosol particles such as PM$_{10}$ and PM$_{2.5}$ \cite{VanLoon2010}. In Stockholm, traffic is a major source of local air pollution, and though air quality is generally good, some streets experience short episodes with severe pollution levels, especially during winter and spring \cite{slbanalys}. %The air in the Stockholm region is regularly monitored by the environmental and health administration (EHA) of the city of Stockholm \cite{slb-matningar}. 

To protect public health, urban air is normally monitored. In addition to monitoring, forecasts of air quality (both hourly and daily) can be critical to regulatory authorities, and in general, there are two approaches to this; with mechanistic models or statistical and/or machine learning models. \cite{ElHarbawi2013, Liao2020, atmos7020015}. %The deterministic dispersion models are common and used e.g. by the EHA of Stockholm \cite{slbanalys}. Some commonly used statistical models include multiple linear regression (MLR), autoregressive integrated moving average (ARIMA), and variants thereof \cite{atmos7020015}. 
With mechanistic models, the processes governing the evolution of air pollution is modeled mathematically, whereas statistical and machine learning models are more data-driven \cite{atmos7020015}. 

From a statistical perspective, predicting air pollution is a time series regression problem, and there are many different regression techniques for forecasting and time series analysis \cite{atmos7020015}. These techniques can vary in complexity, from more simple linear models to deep neural networks capable of finding complex non-linear relationships in the data \cite{atmos7020015, LeCun2015}. Nonetheless, one of the main challenges with air pollution is that there are dependencies over both space and time (i.e., the data is spatio-temporal), and simpler models may not capture these dependencies \cite{Liao2020}. \textcolor{red}{Recent advances in machine learning however have shown promising results when it comes to air quality forecasts, especially deep neural networks \cite{Liao2020, atmos7020015}.} 

%Some variants of combined deep neural networks with one spatial and one temporal component have been implemented however, and these "hybrid networks" have shown promising results \cite{Liao2020, Gilik2021}. 

%However, many common statistical and machine learning techniques have some limitations when applied to air pollution data. For example, they may not be ideally suited when there are dependencies over time, i.e, when the data is sequential (or so-called time-series data), and they may not capture spatial dependencies between different monitoring stations  \cite{Liao2020}. However, with the recent advances in deep learning \cite{LeCun2015}, many of the issues with large, high-dimensional "spatio-temporal" datasets have been  addressed \cite{Liao2020}. Deep learning is capable of detecting intricate input-output relationships in massive datasets, and there are variants of deep neural networks "tailored" to handling both sequential data as well as data where spatial dependencies are important \cite{LeCun2015}. For air pollution forecasts, variants of combined deep neural networks with one spatial and one temporal component have been implemented and these hybrid networks have shown promising results \cite{Liao2020, Gilik2021}. 

%More recently, machine learning methods have become increasingly popular, and some common algorithms for predicting air pollution are random forests, gradient boosting machines, and support vectors machines \cite{FaganeliPucer2018, Arsov2021}. Though successfully implemented, many of the statistical and machine learning models have some limitations when applied to air pollution data. They are not ideally suited when there are dependencies over time, i.e, when the data is sequential (or so-called time-series data), and they may not capture spatial dependencies between different monitoring stations  \cite{Liao2020}. Moreover, the relationship between predictor variables and the response variable(s) in air pollution data may also be very complex and non-linear, and often the data is high-dimensional \cite{Liao2020}. 

%With the recent advances in deep learning \cite{LeCun2015}, many of the issues with large, high-dimensional "spatio-temporal" datasets have been  addressed \cite{Hamdi2021}. Deep learning is capable of detecting intricate input-output relationships in massive datasets, and there are variants of deep neural networks "tailored" to handling both sequential data as well as data where spatial dependencies are important \cite{LeCun2015}. For air pollution forecasts, variants of combined deep neural networks with one spatial and one temporal component have been implemented and these hybrid networks have shown promising results \cite{Liao2020, Gilik2021, Qin2019}. 

%In this work, historical data from several air monitoring stations in the Stockholm region will be used together with weather data to make prediction models of various complexity. The performance of the models will be evaluated and compared to similar work in the literature, but also to the models used by the EHA of Stockholm (mechanistic models). Lastly, the models will be tested on data from other cities, and a transfer learning approach will be used.

\section{Research problem}

Forecasts, be it for weather, stock returns, or future pandemics, are always associated with uncertainty and errors. Erroneous predictions made by existing air pollution forecasting systems, both mechanistic and statistical and/or machine learning-based, can be attributed to many causes. 
In the case of mechanistic models, there can be insufficient information in terms of the factors needed for simulation and modeling \cite{atmos7020015}. For statistical and/or machine learning methods, too simplistic models, lack of data, irrelevant input features, overfitting, etc., can limit prediction accuracy \cite{atmos7020015}. Nevertheless, atmospheric pollution is a very complex phenomenon depending on a multitude of factors across both space and time. Hence, the research problem addressed in this work is:
%\begin{quotation}
%\noindent
\textit{To capture and model the complex dynamics of air pollution with modern machine learning methods, with an emphasis on deep learning.}
%\end{quotation}

%\textit{using state-of-the-art machine learning methods, in particular deep learning, to more successfully capture and model the complex dynamics of air pollution.}

%the research problem this work attempts to address is if state of the art machine learning solutions successfully can model these complex dynamics of air pollution. 

%and existing forecasting models/systems for air pollution make erroneous predictions which tend to be largest when pollution levels peak. In the case of mechanistic models, this is due to insufficient information about necessary parameters  and 

\section{Research question}
From a forecasting perspective, of special interest are episodes when pollution levels peak. Generally, this is also when existing forecasting systems tend to give the largest prediction errors \cite{atmos7020015}. Therefore, the research question this thesis tries to answer is:
%\begin{quotation}
%\noindent
\textit{How can machine learning, in particular deep learning, be used to forecast air pollution levels and pollution peaks?}
%\end{quotation}

%\begin{itemize}
%\item How can machine learning, in particular deep learning, be used to forecast air pollution levels and pollution peaks?
%\makebox[\textwidth]{\textit{How can machine learning, in particular deep learning, be used to forecast air pollution levels and pollution peaks?}}
%\end{itemize}

%an additional sub-question related to the main research question is: \textit{Can machine learning models reliably predict peaks in pollution levels?}


%\begin{center} \textit{Can machine learning models reliably predict peaks in pollution levels?}  \end{center}

%How do machine learning, in particular deep learning, perform in comparison with the dispersion models used by the EHA of Stockholm? Can machine learning algorithms trained on Stockholm data be successfully transferred to other cities with similar conditions? Can density predictions instead of point-predictions be constructed so that if thresholds are of interest, the probability of air pollution levels exceeding a certain level can be calculated?  

\section{Delimitations}

%In this work, historical air pollution and weather data is used. Therefore, the models cannot be tested in "operational mode", i.e., with real-time data to make predictions. Moreover, when forecasting air pollution (hours or days ahead), weather forecasts are often utilized in addition to monitoring data to improve forecast results. Again, with historical data, incorporating weather forecasts is not possible, and consequently the models make predictions based only on multivariate time-series of past observations. This also puts a limit on the time-horizon for the predictions, since without weather forecasts, medium to long-term forecasts of air quality would have large uncertainties.  

%\textcolor{red}{pollution background levels...}


\chapter{Extended Background}
\section{Ambient air pollution}

Ambient air pollution is one of the greatest environmental and health concerns of the modern world. Worldwide, poor air quality causes millions of premature deaths every year and is linked to several adverse health effects such as respiratory problems, cardiovascular disease, and cancer \cite{who2016}. In addition to health risks, the global economic impacts are substantial due to lost labor productivity, increased health care costs, reduced crop yields, etc.\ \cite{oecd2016}. Outdoor air pollution has become a ubiquitous problem, affecting both cities and rural areas, and it is estimated that about 90\% of the world's population are living in regions where air pollution levels exceed guidelines set by the World Health Organization \cite{who2016}. 

\subsection{Principal air pollutants}
\label{sec:airpollutants}
In densely populated urban areas, air pollution levels can periodically be severe, and with an accelerating urbanization, it has become imperative for regulatory authorities to closely monitor city air and try to mitigate the harmful effects of pollution. Commonly monitored substances include sulphur dioxide (SO$_2$), nitrogen oxides (NO$_x$, i.e., NO and NO$_2$), carbon monoxide (CO), ground-level ozone (O$_3$), volatile organic compounds (VOCs), and particulate matter (PM) \cite{VanLoon2010}. 

Vehicular traffic is a major source of the gaseous pollutants NO$_x$, SO$_2$, CO, and VOCs, but certain industrial processes also contribute to emissions \cite{VanLoon2010}. Ground-level O$_3$ (also a gas) is a so-called secondary pollutant that forms when NO$_x$ and VOCs react on sunny days with little wind \cite{VanLoon2010}. 

PM -- the group of pollutants being the focus of this work -- are atmospheric aerosol particles (i.e., particles suspended in the air). They have diverse origins, both natural and anthropogenic, and a complex chemical composition consisting of both solid and liquid species \cite{Schwarzenbach2016}. Some important sources of PM are forest fires, volcanic eruptions, sand/dust storms, sea spray, vehicular traffic, certain industrial processes, construction sites, and domestic combustion \cite{Querol2004, Schwarzenbach2016}. When entering the atmosphere directly by these routes, one denotes the PM as primary. However, PM can also be formed by the oxidation of gases such as SO$_2$, NO$_x$, and VOCs (followed by a complex chemical reaction process), in which case the PM is said to be secondary \cite{Schwarzenbach2016}. PM is also categorized by particle size (or more specifically, the aerodynamic diameter), and particles measuring smaller than 2.5 \textmugreek m and 10 \textmugreek m are denoted as PM$_{2.5}$ and PM$_{10}$, respectively \cite{Schwarzenbach2016}.

Both PM$_{10}$ and PM$_{2.5}$ can travel long distances from point sources (though PM$_{2.5}$ has a longer residence time in the atmosphere than PM$_{10}$), and local pollution can be affected by regional background levels \cite{Schwarzenbach2016, slbanalys}. PM levels are also dependent on weather conditions \cite{Schwarzenbach2016}. For example, temperature and solar radiation are related to the formation of secondary PM, and PM emission from roads, tires, brake wear, etc., can be affected by precipitation and humidity \cite{slbanalys, atmos7020015}. Both PM$_{10}$ and PM$_{2.5}$ are hazardous and cause a wide range of health problems, though PM$_{2.5}$ can more easily penetrate the lungs \cite{Schwarzenbach2016}. In the European Union, annual mean limits are set to 40 \textmugreek g/m$^3$ for PM$_{10}$ and 20 \textmugreek g/m$^3$ for PM$_{2.5}$ \cite{eu-airquality}. 


\subsection{Ambient air pollution in Stockholm}
\label{air-pollution-stockholm}

In the city of Stockholm, environmental air quality standards are usually met, though some streets experience occasional episodes with severe pollution levels (e.g. Hornsgatan is one such street) \cite{slbanalys2021}. Since Stockholm has centralized district heating and few industries, the major source of local CO, NO$_x$, and PM pollution is vehicular traffic \cite{slbanalys, slbanalys2021}. Mechanical wear by studded tires on asphalt and the wearing of brakes and tiers in motor vehicles contribute substantially to local levels of both PM$_{10}$ and PM$_{2.5}$. For PM$_{2.5}$ however, contribution from sources outside of Stockholm is also significant \cite{slbanalys2021}. Emission of SO$_2$ can come from the energy sector and waterborne transport, though local levels are also affected by outside sources. 
%The levels of SO$_2$ are affected by transport from outside sources, though local and regional emissions can be due to the energy sector and waterborne transport \cite{slbanalys2021}. 
For O$_3$, long-range transport from mainland Europe is the single-most important factor contributing to locally measured levels \cite{slbanalys2021}. 

The air in Stockholm County is monitored by Stockholms Luft- och Bulleranalys (SLB-analys), a unit in the Environment and Health Administration (EHA) of the city of Stockholm. SLB-analys are responsible for a number of monitoring stations measuring several air pollutants and some meteorological parameters in the Stockholm region, as well as a few stations outside of Stockholm \cite{slb-matningar}. In addition to monitoring the air, SLB-analys also model and forecast air pollution levels for the Stockholm metropolitan area, and their forecasts are available through a smartphone application, called "Luft i Stockholm" \cite{slbanalys}. 

\section{Forecasting air pollution}

Having the possibility to forecast air pollution levels hours or days ahead can be extremely valuable to regulatory authorities in order to protect public health, and vulnerable groups in particular. In general, there are two broad categories of models for such forecasts; mechanistic models, and statistical and/or machine learning models \cite{ElHarbawi2013}. This work is concerned with the latter type, and in the sections below a review follows. The mathematical and statistical theory behind many of the models is quite extensive \cite{Hastie2009, Montgomery2015, smlbook, LeCun2015}, but relevant theory will be covered briefly.

\subsection{Forecasting as a regression problem}
\label{sec:forecasting}
While mechanistic models are based on mathematical modelling of atmospheric processes along with other factors governing the distribution of air pollution (such as emission source characteristics, physico-chemical properties of pollutants, terrain and building design, etc.) statistical and/or machine learning models are entirely data-driven, being derived directly from measurements on the variables of interest \cite{ElHarbawi2013}. From a statistical (or machine learning) perspective, forecasting air pollution can be viewed as a regression problem, in which a function $f$, mapping input data to a numerical output, is being approximated (or learned) from a training set of labeled input-output examples \cite{smlbook}. Learning the function $f$ amounts to finding a set of parameters (or weights) for the model, which in the case of a simpler regression technique can be only a handful, but possibly millions if a deep neural network is used \cite{smlbook}. Generally in regression, the weights are learned by minimizing a cost function
\begin{align}
J(\bm{\hat{\beta}}) = \frac{1}{n} \sum_{i=1}^{n} \big(\hat{y}_i - y_i \big)^2
\label{eq:cost}
\end{align}
where $\bm{\hat{\beta}}$ is the vector of estimated model parameters ($\hat{\beta}_0, \: \hat{\beta}_1, ..., \: \hat{\beta}_n$), $\hat{y}_i$ is a prediction and $y_i$ is a training data value\cite{smlbook}. In Eq.\ \ref{eq:cost} the squared error loss is used as loss function, and the cost is simply the loss averaged over the training data.\footnote{What is meant by cost and loss functions can vary slightly in the literature, but in this work, the terminology of Lindholm et al. \cite{smlbook} is adopted.} Depending on the model, minimizing $J(\bm{\hat{\beta}})$ is approached differently, as explained further in the sections below. 

\subsection{Linear models}
%\paragraph{Multiple linear regression}
From the wealth of available regression techniques, multiple linear regression (MLR) has been extensively used to forecast and model air pollution \cite{atmos7020015}. If none of the basic model assumptions are violated (i.e., linearity, independence, normality, and constant variance), MLR is often a straightforward method, especially for data with no temporal dependencies (so-called cross-sectional data). However, for time series data, the assumption of independent errors is often not approptiate \cite{Montgomery2015}. 

If fitting a MLR model to time series data, successive errors will typically be correlated (often referred to as autocorrelation), and this will cause several problems with the model if the correlation is not accounted for \cite{Montgomery2015}. To this end, adjustments to the MLR model can be made, some of which require other parameter estimation techniques than the usual least squares method (see below). However, a simple and commonly used procedure to get rid of the autocorrelation is to include one or more lagged values of the response variable as predictors. For example, if the value of the response variable at lag one is included, the MLR model will have the form 
\begin{align}
y_t = \beta_0 + \beta_{1} y_{t-1} + \beta_2 x_{2,t} + ... + \beta_{k} x_{k,t} + \varepsilon_t, \: \: \: t = 1, 2, ..., T
\label{eq:mlr}
\end{align}
where the error term $\varepsilon_t \sim  N(0, \sigma^2)$, and $t$ denotes time steps \cite{Montgomery2015}. The model in Eq.\ \ref{eq:mlr} can be fit with the method of least squares, which in linear regression is the standard way of finding parameters so that $J(\bm{\hat{\beta}})$ is minimized \cite{smlbook}. This is done by solving the so-called normal equations
\begin{align}
(\bm{X}^T\bm{X})\bm{\hat{\beta}} = \bm{X}^T\bm{y}
\label{eq:normal_eq}
\end{align}
and the least squares estimates of the model parameters will then be given by Eq.\ \ref{eq:normal_sol} below (provided that the inverse of $\bm{X}^T\bm{X}$ exists) \cite{smlbook}.
\begin{align}
\bm{\hat{\beta}} = (\bm{X}^T\bm{X})^{-1}\bm{X}^T\bm{y}
\label{eq:normal_sol} 
\end{align}

Careful variable selection in regression is crucial as it can influence the performance of a model. In situations with several variables, one is often concerned with finding an optimal ”subset” of predictors, where multicollinearity should also not be an issue \cite{Montgomery2012}. To this end, variable selection techniques based on optimizing a criterion like the Akaike or Bayes information criterion are common, and typically multicollinearity is also tested for \cite{Montgomery2012}. However, if one is reluctant to exclude variables, but multicollinearity still might be an issue, regularized versions of MLR can be used \cite{smlbook, Montgomery2012}. 

Two common techniques are $L_1$ and  $L_2$ regularization, in which an extra so-called "penalty" term is added to the cost function to shrink the estimated model parameters. In $L_2$ regularization (also called ridge regression), the parameters will be pushed towards small values, whereas in $L_1$ regularization (or lasso regression), some parameters will be driven to zero. The penalty terms for ridge and lasso regression are, respectively, 
$$\lambda \sum_{j=1}^{k}\beta_{j}^2\:\:\: \text{and}\:\:\: \lambda \sum_{j=1}^{k}|\beta_{j}|$$
where $\lambda$ is a parameter controlling the shrinkage \cite{smlbook}. For ridge regression, the parameter estimates can be found by solving a modified version of Eq.\ \ref{eq:normal_eq}, while for lasso regression, no such analytical solution exists, and numerical optimization techniques have to be used instead \cite{smlbook}. By shrinking the parameters, ridge and lasso regression works as a variable selection method, while also preventing overfitting when used in more complex regression models \cite{smlbook}.

The extensive use of MLR for air pollution forecasts is many times motivated by its simplicity and straightforward implementation \cite{atmos7020015}. Another advantage is interpretability; for example, inference can be made on all input variables, allowing one to investigate their individual importance \cite{Montgomery2012}. However, the assumption of linearity might not always hold, and rather large prediction errors have been observed at times of pollution peaks \cite{atmos7020015}. Moreover, with data from several (but nearby) monitoring stations, collinearity can be an issue, which is why ridge or lasso regression are popular alternatives to the more classical non-regularized MLR model \cite{FaganeliPucer2018}. 

%Linear models for time-series analysis, such as auto-regressive moving average and auto-regressive \textit{integrated} moving average (ARMA and ARIMA, respectively) and variants thereof, are also common \cite{Arsov2021, Goyal2006}. These models make predictions of future values based on past data (i.e., taking dependencies over time into account), however, similar to MLR, linearity is assumed and errors can be large when there are temporary peaks in pollution levels \cite{atmos7020015}. 

%\paragraph{Bayesian methods}
%While the forecasting techniques discussed above produce point-predictions of a pollutant, Bayesian methods can be used to predict distributions (or put another way, make density forecasts) \cite{FaganeliPucer2018}. A Bayesian approach can offer some advantages if thresholds are of interest, since with a density forecast, the probability of pollution levels exceeding a certain value can be estimated \cite{FaganeliPucer2018, smlbook}. For example, in Pucer et al. \cite{FaganeliPucer2018}, a Gaussian process model was used to give Gaussian density predictions of PM$_{10}$ and O$_3$. Bayesian methods have also been used for spatial predictions of PM by spatial interpolation (using values from monitored locations to estimate levels at other locations without any monitoring) \cite{atmos7020015}. 

\subsection{Extensions of the linear model}
More versatile and flexible regression models generally produce better forecasting results than linear techniques \cite{atmos7020015}. Some examples include regression trees, generalized additive models, and support vector machines (SVM) \cite{atmos7020015, FaganeliPucer2018}. These models can handle more complex non-linear input-output relationships, and especially SVM has been successfully applied for PM$_{10}$ prediction, sometimes with better results than artificial neural networks \cite{atmos7020015}. 

Artificial neural networks (ANNs), in particular the multilayer perceptron (MLP), have also been extensively used as a forecasting technique \cite{atmos7020015}. ANNs are flexible models able to handle non-linear input-output relationships, however, over-fitting can be an issue, especially with high-dimensional input and if training data is limited \cite{atmos7020015, FaganeliPucer2018}. 

%The MLP is a so-called feedforward neural network, in which a set of inputs are taken, passed through several layers of so-called hidden units, eventually producing an output \cite{LeCun2015}. 
The MLP is a so-called feedforward neural network, in which a set of input data is taken and passed through several "hidden" layers made up of neurons (also called units), before an output is produced \cite{LeCun2015}.
%A common way to illustrate a MLP is given in Figure \ref{fig:ANN}, where a network consisting of the input layer, two hidden layers (where units are represented with circles), and a single output, is shown. 
Deep neural networks can have many such layers (hence the term "deep" \cite{Chollet2017}), and each layer can have hundreds of units. Every layer produces a slightly more abstract representation of its input by non-linear transformations, and with several such transformations, complex relationships in the data can be learned \cite{LeCun2015}.

%\begin{figure}[h]
%\begin{center}
%\includegraphics{neural-network}
%\caption{Artificial neural network with two hidden layers.}
%\label{fig:ANN}
%\end{center}
%\end{figure}

%Training the neural network is an iterative process, in which the weights (or parameters) of the network are adjusted until the measured error stops decreasing.

Many other deep learning architectures than the MLP exist, such as convolutional neural networks (CNNs), or recurrent neural networks (RNNs). CNNs are commonly used for image recognition while RNNs (and variants thereof) normally are applied to sequential data. 
%\textcolor{red}{(This section will be expanded with some more theory for the deep learning models to be used. Also, some mathematical notation will added.)}

\subsection{Variable selection}

In any regression problem, variable/feature selection is crucial as it can influence the performance of a model. 
%, and rarely are all available input features necessary or even desirable to include (as some might worsen performance) \cite{smlbook}. 
%As pointed out in section \ref{sec:airpollutants}, 
With regards to PM, as pointed out in section \ref{sec:airpollutants}, weather conditions can greatly affect pollution levels, and therefore meteorological data can be utilized to improve forecasts \cite{atmos7020015}.

% ### CORRELATION PLOT ###
\begin{figure}[h]
\makebox[\textwidth][c]{\includegraphics[width=0.85\textwidth]{/Users/simoncarlen/desktop/luftdata/plots/Torkel Knutssongatan_correlation}}
\caption{Pairwise correlations between air pollutants and some meteorological variables.}
\label{fig:correlationplot}
\end{figure}
% #######################

Additional variables can also be included in PM forecasts. For example, motor traffic data such as travel speeds, traffic flow and intensity, etc., can be utilized \cite{atmos7020015}. Data on other pollutants can also be important, especially SO$_2$ and NO$_x$ as they are involved in the formation of secondary PM \cite{Arsov2021}. Moreover, if forecasts focus solely on PM$_{10}$ (as in this work), data on PM$_{2.5}$ can further improve the results  \cite{Arsov2021}. Temporal variables such as time of the day and time of year are also useful since daily and seasonal variation of PM pollution is important \cite{Schwarzenbach2016, atmos7020015}. 
%The selection and preprocessing of variables in this work is described in detail in section \ref{chap:dataprocesschap}. 
%A detailed description of all variables utilized in this work (and how they were processed), is given in section \ref{chap:dataprocesschap}.

In Figure \ref{fig:correlationplot} where pairwise correlations between a few meteorological variables and PM$_{10}$ at different stations in the Stockholm region are given (see Table \ref{tab:stations} for details about the different monitoring stations), it can be inferred that PM$_{10}$ correlate negatively with humidity, but positively with atmospheric pressure and solar radiation. It can also be seen that PM$_{10}$ levels are strongly correlated among some stations. 

In this work, in addition to PM$_{10}$ data, meteorological data as well as data on PM$_{2.5}$ and NO$_x$ were utilized. Some features used as input to the models were also derived. A more detailed description of the variables and their preprocessing is given in section \ref{chap:dataprocesschap}.

\section{Summary and motivation for this work}

\chapter{Methodology}
%General implementation of the stategy here before going into details about sources, preprocessing, hyperparameter tuning etc, ... Perhaps also a pic?
The major steps of the implemented workflow were as follows; 

Detailed descriptions of each step in the process are given in subsequent sections

%is shown in Figure \ref{fig:dataflow}. Historical air pollution data from several monitoring stations, together with meteorological data from one station, was retrieved, preprocessed (with some features engineered), and divided into data windows. Three deep learning models (feed forward neural network, RNN, and LSTM) were trained and tested for short-term predictions (one hour ahead) of PM$_{10}$ for one station at Torkel Knutssongatan (measuring urban background levels, see Table \ref{tab:stations}). As baseline models for comparison, multiple linear regression and ARIMA were used. Detailed descriptions of each step in the process are given in subsequent sections. 

%\begin{figure}[htbp]
%\begin{center}
%\makebox[\textwidth][c]{\includegraphics[width=1\textwidth]{workflow}}
%\caption{Implemented workflow.}
%\label{fig:dataflow}
%\end{center}
%\end{figure}

\section{Data retrieval and preprocessing}
\label{chap:dataprocesschap}

\subsection{Data sources}
\label{sec:data-sources}

Air pollution data was retrieved from the Swedish Meteorological and Hydrological Institute's (SMHI) centralized database for air quality measurements \cite{smhi-luftmatningar}. This data is part of the national and regional environmental monitoring of Sweden, a program coordinated and funded by the Swedish Environmental Protection Agency (Swedish EPA) and the Swedish Agency for Marine and Water Management. There are in total ten different program areas, of which air is one, and all data are licensed under CC0 and therefore freely accessible to the public \cite{naturvardsverket-miljodata}. For the national air monitoring (under Swedish EPA's responsibility), SMHI acts as a national data host and stores (quality checked) historical data reported yearly from municipalities in Sweden \cite{smhi-luftmatningar}.

\subsubsection{Monitoring stations}
In Stockholm County, there are 19 stationary sites for air pollution monitoring \cite{slb-matningar}, and initially, data from each site was considered. However, many stations have irregular data series, and not all stations measure the same set of pollutants. Due to this, data from three sites with hourly measurements of PM$_{10}$ and PM$_{2.5}$ (in \textmugreek g/m$^3$) for the time period 2016-01-01 to 2022-01-01 was chosen, giving a total of 52,609 data points. For the station at which PM predictions subsequently were to be made (Torkel Knutssonsgatan), hourly data of NO$_2$ was also included. As described in section \ref{air-pollution-stockholm}, SLB-analys also monitor several weather parameters, and hourly measurements of temperature (in $^\circ$C), precipitation (mm), atmospheric pressure (hPa), relative humidity (as \%), solar radiation (W/m$^2$), and wind speed (m/s) were also included from the station at Torkel Knutssonsgatan. The meteorological data were downloaded from SLB-analys' webpage \cite{slb-analys-meteorologi}. In general, air pollution monitoring can be classified by the surrounding area (rural, rural-regional, rural-remote, suburban, and urban), and by the predominant emission sources (background, industrial, or traffic) \cite{smhi-luftmatningar}. The chosen stations included data from both traffic and background monitoring, in urban as well as rural-regional areas. More information about the stations are given in Table \ref{tab:stations} in appendix \ref{chapt:appendix_A}. 

%% STATIONS TABLE
%% ##############
%\begin{table}[]
%\centering
%\caption{Monitoring stations in Stockholm County.}
%\label{tab:stations}
%\resizebox{\textwidth}{!}{%
%\begin{tabular}{@{}llllll@{}}
%\toprule
%Station                         & Station code & Longitude & Latitude  & Type of monitoring                                                   & Parameters                                                                                             \\ \midrule
%Norrtälje, Norr Malma           & 18643        & 18.631313 & 59.832382 & \begin{tabular}[c]{@{}l@{}}Rural-Regional \\ Background\end{tabular} & PM$_{10}$, PM$_{2.5}$                                                                                  \\ \midrule
%Stockholm, Hornsgatan 108       & 8780         & 18.04866  & 59.317223 & Urban Traffic                                                        & PM$_{10}$, PM$_{2.5}$                                                                                  \\ \midrule
%Stockholm, Torkel Knutssonsgatan & 8781         & 18.057808 & 59.316006 & Urban background                                                     & \begin{tabular}[c]{@{}l@{}}PM$_{10}$, PM$_{2.5}$, NO$_2$, \\ meteorological \\ parameters\end{tabular} \\ \bottomrule
%\end{tabular}%
%}
%\end{table}
%% ##############

\subsection{Data preprocessing}
\subsubsection{Initial preprocessig}
Time series plots for PM$_{10}$ and PM$_{2.5}$ at Torkel Knutssonsgatan are shown in \vref{fig:time_series_plots}. (Similar plots but for all stations are given in Fig.\ \ref{fig:time_series_plots_all} in appendix \ref{chapt:appendix_A}.)
%(station-wise, with PM$_{10}$ and PM$_{2.5}$ in the left and right subplots, respectively)
Some stations had short episodes with missing data, and linear interpolation was used to fill in the missing values. However, for the PM$_{2.5}$ data at Torkel Knutssonsgatan (plot (b) in \vref{fig:time_series_plots}), due to the rather big gap at the beginning of 2019, a train-test split was done to entirely avoid this period (see below). Missing weather data was also linearly interpolated, except for the variables atmospheric pressure and wind speed for which mean imputation was deemed more appropriate. 
%Moreover, before use in any of the models, all data were min-max normalized (i.e., scaled to the interval $[0,1]$). 

It should be noted that some PM values were negative.
%this is seen clearly in e.g.\ plot (f) in \vref{fig:time_series_plots}, for the time period Jan.\ 2016 up to about Sep.\ 2019. 
However, negative values are expected since automated measuring instruments for PM (due to noise) may produce values between zero and the negative detection limit, especially when there are rapid changes in humidity (SMHI, personal communication, April 11, 2022). These values are therefore not to be considered any more "incorrect" than positive values, though it may at first seem contradictory to include them in an analysis. 

\begin{figure}[h]
\centering
\makebox[\textwidth][c]{\includegraphics[width=1\textwidth]{../plots/time_series_plots_target_station}}
%\includegraphics[width=\textwidth]{../plots/time_series_plots}
\caption{Time series plots for (a) PM$_{10}$ and (b) PM$_{2.5}$ at Torkel Knutssonsgatan.}
\label{fig:time_series_plots}
\end{figure}

%\begin{figure}[h]
%\centering
%\makebox[\textwidth][c]{\includegraphics[width=0.9\textwidth]{../plots/time_series_plots_7}}
%\caption{Time series plots for PM$_{10}$ and PM$_{2.5}$.}
%\label{fig:time_series_plots}
%\end{figure}

%\paragraph{Feature engineering}
%From the meteorological data, wind vectors ($u$ and $v$) were derived from wind direction and wind speed, as wind vectors are more suitable model inputs \cite{tensorflow-timeseries}. After converting wind direction values to radians, $u$ and $v$ were obtained in the following way $$ u = ws * cos(\theta)$$ $$v = ws * sin(\theta)$$ where $ws$ denote wind speed and $\theta$ is the wind direction (in radians). 
\subsubsection{Feature engineering}
In \cref{fig:time_series_plots}, yearly periodicity in the data can be seen, especially for PM$_{10}$ where levels tend to be higher during spring. Daily and weekly periodicity is also expected since traffic intensities vary throughout the day and week. 
%The meteorological variables such as temperature, solar radiation, etc.\ also have periodicity. 
To account for this, timestamps were converted to temporal variables as sine and cosine signals for day, week, and year. For example, the sine and cosine signals for day were calculated in the following way

$$ \text{Sine day = sin} \Big (\text{timestamp} \cdot \frac{2\pi}{86,400} \Big)$$
$$ \text{Cosine day = cos} \Big (\text{timestamp} \cdot \frac{2\pi}{86,400} \Big)$$
where timestamp is in seconds (and with 86,400 seconds in 24 hours, dividing by this term is necessary). The calculations were done similarly for week and year, except for the term in the denominator which instead was set to seconds per week and seconds per year, respectively. 
%The transformations were done so that the sine and cosine functions oscillate between zero and one. 
The temporal variables for day in a 24 hour time window are shown in \vref{fig:time_sine_cos}.

\begin{figure}[h] 
\begin{center}
%\makebox[\textwidth][c]{\includegraphics[width=.585\textwidth]{../plots/time_signals}}
\includegraphics[scale=1.05]{../plots/time_signals}
\caption{Temporal variables for day as sine and cosine signals.}
\label{fig:time_sine_cos}
\end{center}
\end{figure}

\paragraph{Sliding windows}
Sliding windows from the data were also created. The sliding window approach is used for time-series forecasting where windows (or sequences of certain lengths, also called frames) are extracted from the input data \cite{Arsov2021, Gilik2021}. In each window, there are two "sub-windows"; the input window and the target window, and the target window is offset by some amount of time from the input window. For example, as shown in Figure \ref{fig:sliding-window}, the total window length is nine time steps, and the first eight time steps is the input window used to predict the target window (in this case having a length of one) one time step in the future. After extracting a data sequence, the window slides to the right one (or more) steps and extracts the next sequence. This is continued until time step $n$ at which point all the data have been processed. 
In this work, input windows of different lengths were tested to make short-term predictions for a target window with a length of one (more details are given in section \ref{sec:tuning} below). 

 \begin{figure}[h]
\begin{center}
\includegraphics{sliding-windows}
\caption{Sliding window approach for time-series data.}
\label{fig:sliding-window}
\end{center}
\end{figure}

%In this work, input window lengths were set to 12 hr, and predictions were made 1 hr, 6 hr, and 12 hr in the future (giving total window lengths of 13 hr, 18 hr, and 24 hr, respectively). Moreover, single-output models predicting PM levels at one urban background station (Torkel Knutssongatan),  as well as multi-output models predicting PM levels at all urban traffic stations, were tested.

%In this work, the window lengths were set to 12 hours, and both single time-step predictions (target windows of length one) and multi-time-step predictions (target windows of length $n$) were tested. Moreover, prediction models giving both single-outputs (the target value at one monitoring station) as well as multi-outputs (target value at all stations) were constructed.

%\paragraph{Train-test split} Lastly, the data was split into training (60\%), validation (20\%), and test (20\%) sets, where the validation set was used for hyperparameter tuning (described in more detail in section \ref{sec:tuning}). Being sequence data, the sampling was done consecutively, without random shuffling, so that order information would be preserved \cite{Gilik2021}. 
\paragraph{Train-test split} Lastly, the data was split into training, validation, and test sets, where the validation set was used for hyperparameter optimization. The test set was taken as the most recent year of data (from 2019-09-16 to 2020-09-16), the validation set was taken as the year prior to the test data (2018-09-16 to 2019-09-16), and the remaining data was used for training (2016-01-01 to 2018-09-16). This split is motivated by the fact that the data is in the form of time-series, where each observation has a specific time-stamp and where successive observations are (in this case) positively autocorrelated.

%Being sequence data, the sampling was done consecutively, without random shuffling, so that order information would be preserved 

%/Users/simoncarlen/desktop/thesis/data_and_code/plots/PM10_2016_2020_2

%\begin{figure}[h]
%\centering
%\makebox[\textwidth][c]{\includegraphics[width=1.1\textwidth]{../plots/time_series_plots}}
%\caption{Time series plots for PM$_{10}$ and PM$_{2.5}$.}
%\label{fig:time_series_plots}
%\end{figure}

\section{Hyperparameter tuning and model fitting}
\label{sec:tuning}

\subsection{Multiple linear regression models}
Initially, a simple linear regression model was fit with OLS where a value of the response variable at lag one was used as predictor. This model did not display any autocorrelation, but when also including the response variable at lag two as predictor, the Durbin-Watson statistic improved (i.e., was brought closer to 2). Including additional response variables after the first two lags did not lead to further improvements in terms of eliminating autocorrelation. It should be noted that a log transformation of the variables were required to stabilize the variance and also bring the residuals closer to a normal distribution. Even so, deviation from normality was indicated, as can be seen in the residual plots in \cref{fig:residuals_MLR_PM10} in appendix \ref{chapt:appendix_B}. Generally, this is less of a problem if prediction (and not inference) is the sole purpose of a regression analysis, and the central limit theorem also ensures that confidence intervals, t-tests, etc.\ will be increasingly accurate as the sample size increases, even if the distribution of the residuals is not normal \cite{Montgomery2012}. 

Including data (also log transformed) from other stations improved the MLR model, as did including the temporal variables for day, but the temporal variables for week and year were not significant ($p$-values $>$ 0.05). Data from other stations were, similar as for the response variable, entered into the model as the values at lag one (since the values at time $t+1$ cannot be known). Despite having data from several nearby stations, the condition number did not indicate strong multicollinearity. However, when also including weather parameters in the model, the condition number rose considerably. 

% lagged response variables as predictors was fit with OLS. The Durbin-Watson test showed that for both PM$_{10}$ and PM$_{2.5}$, including one response variable ($y_{t-1}$ and $y_{t-2}$) was enough to eliminate any autocorrelation, but 
%With data from three stations in the Stockholm area, some collinearity was expected, and regularized MLR models were initially tried. However, 

\subsection{Deep learning models}

%The Keras Tuner library \cite{omalley2019kerastuner} was used to find the best set of hyperparameters for each model (except for the MLR and ARIMA models used as baseline). 
%%Hyperparameters at both the model architecture-level as well as the input data-level were included in the search space. More specifically, at the input data-level, the width of the windows were set to 3 h, 6 h, or 12 h, and the models were fit with the different versions of the input data. 
%More specifically, the following hyperparameters were tuned:
%
%\begin{itemize}
%\item Number of layers (up to five were tested)
%\item Number of units per layer (in the range [32, 512] with step size set to 32)
%\item Learning rate (sampled uniformly in the range [0.0001, 0.01])
%\item Number of epochs 
%\end{itemize} 
%
%%For some models, a dropout layer (with rates in the range [0, 0.3] and step size set to 0.05) was also tested.
%For the hyperparameter search, Bayesian optimization was used as tuner. (The Bayesian optimization tuner tries to predict which hyperparameters that are likely to improve the model given previous results \cite{omalley2019kerastuner}). The motivation for this choice is the large number of possible hyperparameter combinations, making it infeasible to test all of them within a reasonable amount of time. Instead, it was assumed that the tuner after 75 trials would find some optimal set of hyperparameters. 
%The hyperparameter search was done in total three times for every model; one search each was performed for data input windows of different sizes, namely 8 h, 16 h, and 24 h. After completing the search, the number of epochs for each model were tuned, and all models were re-trained and evaluated on the validation and test data. 

%\begin{figure}[h]
%\centering
%\makebox[\textwidth][c]{\includegraphics[width=0.9\textwidth]{../plots/time_series_plots_7}}
%\caption{Time series plots for PM$_{10}$ and PM$_{2.5}$.}
%\label{fig:time_series_plots}
%\end{figure}

%, the models were re-trained on the training set plus the validation set, and performance on the test set was recorded. Again, with three different window sizes tested, three versions per model were obstained. All predictions were made for PM$_{10}$ one hour ahead for the station at Torkel Knutssongatan (measuring urban background levels in the center of Stockholm).


\chapter{Results and Discussion}
\section{Multiple linear regression models}


\chapter{Conclusions}
% MLR model simple, but lots of data prep

% deep learning models not as much data prep (and checking model assumptions), but longer to train

% autoregressive version could be tried, but would require predicting/forecasting NO2 values att ALL stations

% NO2 also modeled well with a simple linear techniqe sometimes a simpler model can do the job....
% variance increase as levels increase... not good for MLR? make stattionary with ARIMA models a better idea?
% confodence indervals
% additional several steps forecasts..

In this work, several different deep neural network architectures have been explored and compared with a multiple linear regression model for predicting hourly urban background levels of NO$_2$ using time series data. Generally, multiple linear regression models are straightforward to implement compared to deep neural networks, though additional data preparation steps were necessary here, as several of the model assumptions were violated. The deep neural network models required considerably less data preparation, but took a substantial amount of time to train and tune compared to the multiple linear regression model. 

Across several evaluation metrics, the deep neural network models performed better than the multiple linear regression model. Particularly, a recurrent neural network model (the LSTM model) were consistently better than the rest of the models. The theoretical performance of this model should also reflect performance in operational mode well. However, sudden NO$_2$ peaks were poorly predicted, and generally for all models, forecasts at high NO$_2$ values were unsatisfactory. Moreover, none of the models had the desired structure of the forecasts errors, and this warrants further model refinements. 


%The forecast horizons here are very short-term (one hour), which of course limits usability. All models in this work could however quite easily be adapted to forecast horizons of arbitrary length by utilizing the predictions made (i.e., by plugging them back into the models). 

 
% modelling other pollutatns entierly possible
% hourly forecasts are valuable as they give 


% references section
\bibliography{references}{}
\bibliographystyle{ieeetr}
%\bibliographystyle{agsm}

%\appendix
%\chapter{Monitoring Stations}
%\label{chapt:appendix_A}
% STATIONS TABLE
% ##############
Information about the monitoring stations from which data was used is summarized in \vref{tab:stations}. In \vref{fig:time_series_plots_all}, time series plots of PM$_{10}$ and PM$_{2.5}$ at each station are shown. 
\begin{table}[h]
\centering
\caption{Monitoring stations.}
\label{tab:stations}
\resizebox{\textwidth}{!}{%
\begin{tabular}{@{}llllll@{}}
\toprule
Station                         & Station code & Longitude & Latitude  & Type of monitoring                                                   & Parameters                                                                                             \\ \midrule
Norrtälje, Norr Malma           & 18643        & 18.631313 & 59.832382 & \begin{tabular}[c]{@{}l@{}}Rural-Regional \\ Background\end{tabular} & PM$_{10}$, PM$_{2.5}$                                                                                  \\ \midrule
Stockholm, Hornsgatan 108       & 8780         & 18.04866  & 59.317223 & Urban Traffic                                                        & PM$_{10}$, PM$_{2.5}$                                                                                  \\ \midrule
Stockholm, Torkel Knutssonsgatan & 8781         & 18.057808 & 59.316006 & Urban background                                                     & \begin{tabular}[c]{@{}l@{}}PM$_{10}$, PM$_{2.5}$, NO$_2$, \\ meteorological \\ parameters\end{tabular} \\ \bottomrule
\end{tabular}%
}
\end{table}

%% ##############
%\begin{figure}[h]
%\centering
%\makebox[\textwidth][c]{\includegraphics[width=1\textwidth]{../plots/time_series_plots_all}}
%%\includegraphics[width=\textwidth]{../plots/time_series_plots}
%\caption{Time series plots for PM$_{10}$ and PM$_{2.5}$ at all stations.}
%\label{fig:time_series_plots_all}
%\end{figure}

\pagebreak
\chapter*{Appendices}
\counterwithin{figure}{section}
\counterwithin{table}{section}
%% provide three setup instructions:
\addcontentsline{toc}{chapter}{Appendices} % write to the toc file
\setcounter{section}{0}
\renewcommand\thesection{\Alph{section}}

\section{Monitoring stations}
\label{chapt:appendix_A}
% STATIONS TABLE
% ##############
Information about the monitoring stations from which data was used is summarized in \vref{tab:stations}. In \vref{fig:time_series_plots_all}, time series plots of PM$_{10}$ and PM$_{2.5}$ at each station are shown. 
\begin{table}[h]
\centering
\caption{Monitoring stations.}
\label{tab:stations}
\resizebox{\textwidth}{!}{%
\begin{tabular}{@{}llllll@{}}
\toprule
Station                         & Station code & Longitude & Latitude  & Type of monitoring                                                   & Parameters                                                                                             \\ \midrule
Norrtälje, Norr Malma           & 18643        & 18.631313 & 59.832382 & \begin{tabular}[c]{@{}l@{}}Rural-Regional \\ Background\end{tabular} & PM$_{10}$, PM$_{2.5}$                                                                                  \\ \midrule
Stockholm, Hornsgatan 108       & 8780         & 18.04866  & 59.317223 & Urban Traffic                                                        & PM$_{10}$, PM$_{2.5}$                                                                                  \\ \midrule
Stockholm, Torkel Knutssonsgatan & 8781         & 18.057808 & 59.316006 & Urban background                                                     & \begin{tabular}[c]{@{}l@{}}PM$_{10}$, PM$_{2.5}$, NO$_2$, \\ meteorological \\ parameters\end{tabular} \\ \bottomrule
\end{tabular}%
}
\end{table}

%% ##############
%\begin{figure}[h]
%\centering
%\makebox[\textwidth][c]{\includegraphics[width=1\textwidth]{../plots/time_series_plots_all}}
%%\includegraphics[width=\textwidth]{../plots/time_series_plots}
%\caption{Time series plots for PM$_{10}$ and PM$_{2.5}$ at all stations.}
%\label{fig:time_series_plots_all}
%\end{figure}

\pagebreak
\section{Model diagnostics and summary statistics for the multiple linear regression models}
\label{chapt:appendix_B}
Residual plots from the OLS regression are shown in \cref{fig:residuals_MLR} below. From plot (a) and (c), the long-tailed distribution of the errors can be seen, especially in plot (a) where the long tails are indicated by deviations from the straight line. Looking at plot (b), the variance appears stable, and the residuals are scattered in a reasonably random fashion and there is also no signs of non-linearity. The variance also appear stable over time, as indicated in plot (d). An additional numeric test for constant variance was performed where a regression line was fit to $\sqrt{|\hat{\varepsilon}}|$. This line had a non-significant slope (with $x = -0.0045$ and $p=0.262$), giving further support to the homoscedasticity assumption \cite{Montgomery2012}. In Table \ref{tab:OLS_table} and \ref{tab:Robust_table}, summary statistics are shown for the OLS regression and robust regression, respectively. 

\begin{figure}[h]
\centering
\makebox[\textwidth][c]{\includegraphics[width=.95\textwidth]{../plots/Residual_plots_MLR_NO2}}
\caption{Residual plots for the OLS regression model.}
\label{fig:residuals_MLR}
\end{figure}

\begin{landscape}
\begin{table}
\begin{center}
\begin{tabular}{lclc}
\toprule
\textbf{Dep. Variable:}                                     & NO\$\_2\$, Torkel Knutssonsgatan & \textbf{  R-squared:         } &     0.858   \\
\textbf{Model:}                                             &               OLS                & \textbf{  Adj. R-squared:    } &     0.858   \\
\textbf{Method:}                                            &          Least Squares           & \textbf{  F-statistic:       } & 1.136e+04   \\
\textbf{Date:}                                              &         Fri, 19 Aug 2022         & \textbf{  Prob (F-statistic):} &     0.00    \\
\textbf{Time:}                                              &             12:42:52             & \textbf{  Log-Likelihood:    } &    39712.   \\
\textbf{No. Observations:}                                  &               26281              & \textbf{  AIC:               } & -7.939e+04  \\
\textbf{Df Residuals:}                                      &               26266              & \textbf{  BIC:               } & -7.927e+04  \\
\textbf{Df Model:}                                          &                  14              & \textbf{                     } &             \\
\textbf{Covariance Type:}                                   &            nonrobust             & \textbf{                     } &             \\
\bottomrule
\end{tabular}
\begin{tabular}{lcccccc}
                                                            & \textbf{coef} & \textbf{std err} & \textbf{t} & \textbf{P$> |$t$|$} & \textbf{[0.025} & \textbf{0.975]}  \\
\midrule
\textbf{intercept}                                          &       0.1269  &        0.005     &    23.332  &         0.000        &        0.116    &        0.138     \\
\textbf{NO\$\_2\$, Stockholm Torkel Knutssonsgatan, lag1}   &       0.9222  &        0.006     &   142.835  &         0.000        &        0.910    &        0.935     \\
\textbf{NO\$\_2\$, Stockholm Torkel Knutssonsgatan, lag2}   &      -0.2021  &        0.006     &   -35.796  &         0.000        &       -0.213    &       -0.191     \\
\textbf{NO\$\_2\$, Stockholm Torkel Knutssonsgatan, lag 24} &       0.0512  &        0.003     &    18.420  &         0.000        &        0.046    &        0.057     \\
\textbf{NO\$\_2\$, Stockholm Hornsgatan 108 , lag1}         &       0.0484  &        0.004     &    11.632  &         0.000        &        0.040    &        0.057     \\
\textbf{NO\$\_2\$, Stockholm Sveavägen 59 , lag1}           &      -0.0414  &        0.004     &   -10.076  &         0.000        &       -0.049    &       -0.033     \\
\textbf{NO\$\_2\$, Stockholm E4/E20 Lilla Essingen, lag1}   &       0.1133  &        0.005     &    23.308  &         0.000        &        0.104    &        0.123     \\
\textbf{Sine day}                                           &       0.0017  &        0.001     &     1.370  &         0.171        &       -0.001    &        0.004     \\
\textbf{Cosine day}                                         &      -0.0524  &        0.001     &   -37.946  &         0.000        &       -0.055    &       -0.050     \\
\textbf{Sine week}                                          &      -0.0105  &        0.001     &   -10.986  &         0.000        &       -0.012    &       -0.009     \\
\textbf{Cosine week}                                        &       0.0132  &        0.001     &    13.325  &         0.000        &        0.011    &        0.015     \\
\textbf{Temperature}                                        &      -0.0038  &        0.003     &    -1.501  &         0.133        &       -0.009    &        0.001     \\
\textbf{Relative humidity}                                  &      -0.0188  &        0.002     &    -8.710  &         0.000        &       -0.023    &       -0.015     \\
\textbf{Solar radiation}                                    &      -0.0864  &        0.003     &   -32.090  &         0.000        &       -0.092    &       -0.081     \\
\textbf{Wind speed}                                         &      -0.1473  &        0.004     &   -37.139  &         0.000        &       -0.155    &       -0.140     \\
\bottomrule
\end{tabular}
\begin{tabular}{lclc}
\textbf{Omnibus:}       & 1882.435 & \textbf{  Durbin-Watson:     } &    1.939  \\
\textbf{Prob(Omnibus):} &   0.000  & \textbf{  Jarque-Bera (JB):  } & 6919.847  \\
\textbf{Skew:}          &   0.298  & \textbf{  Prob(JB):          } &     0.00  \\
\textbf{Kurtosis:}      &   5.442  & \textbf{  Cond. No.          } &     58.2  \\
\bottomrule
\end{tabular}
\caption{OLS Regression Results.}
\label{tab:OLS_table}
\end{center}
\end{table}
\end{landscape}


\begin{landscape}
\begin{table}
\begin{center}
\begin{tabular}{lclc}
\toprule
\textbf{Dep. Variable:}                                     & NO\$\_2\$, Torkel Knutssonsgatan & \textbf{  No. Observations:  } &    26281    \\
\textbf{Model:}                                             &               RLM                & \textbf{  Df Residuals:      } &    26266    \\
\textbf{Method:}                                            &               IRLS               & \textbf{  Df Model:          } &       14    \\
\textbf{Norm:}                                              &              HuberT              & \textbf{                     } &             \\
\textbf{Scale Est.:}                                        &               mad                & \textbf{                     } &             \\
\textbf{Cov Type:}                                          &                H1                & \textbf{                     } &             \\
\textbf{Date:}                                              &         Fri, 19 Aug 2022         & \textbf{                     } &             \\
\textbf{Time:}                                              &             12:45:23             & \textbf{                     } &             \\
\textbf{No. Iterations:}                                    &                23                & \textbf{                     } &             \\
\bottomrule
\end{tabular}
\begin{tabular}{lcccccc}
                                                            & \textbf{coef} & \textbf{std err} & \textbf{z} & \textbf{P$> |$z$|$} & \textbf{[0.025} & \textbf{0.975]}  \\
\midrule
\textbf{intercept}                                          &       0.1178  &        0.005     &    24.229  &         0.000        &        0.108    &        0.127     \\
\textbf{NO\$\_2\$, Stockholm Torkel Knutssonsgatan, lag1}   &       0.9596  &        0.006     &   166.213  &         0.000        &        0.948    &        0.971     \\
\textbf{NO\$\_2\$, Stockholm Torkel Knutssonsgatan, lag2}   &      -0.2101  &        0.005     &   -41.615  &         0.000        &       -0.220    &       -0.200     \\
\textbf{NO\$\_2\$, Stockholm Torkel Knutssonsgatan, lag 24} &       0.0453  &        0.002     &    18.244  &         0.000        &        0.040    &        0.050     \\
\textbf{NO\$\_2\$, Stockholm Hornsgatan 108 , lag1}         &       0.0375  &        0.004     &    10.086  &         0.000        &        0.030    &        0.045     \\
\textbf{NO\$\_2\$, Stockholm Sveavägen 59 , lag1}           &      -0.0351  &        0.004     &    -9.571  &         0.000        &       -0.042    &       -0.028     \\
\textbf{NO\$\_2\$, Stockholm E4/E20 Lilla Essingen, lag1}   &       0.1014  &        0.004     &    23.344  &         0.000        &        0.093    &        0.110     \\
\textbf{Sine day}                                           &      -0.0001  &        0.001     &    -0.130  &         0.896        &       -0.002    &        0.002     \\
\textbf{Cosine day}                                         &      -0.0485  &        0.001     &   -39.317  &         0.000        &       -0.051    &       -0.046     \\
\textbf{Sine week}                                          &      -0.0088  &        0.001     &   -10.269  &         0.000        &       -0.010    &       -0.007     \\
\textbf{Cosine week}                                        &       0.0111  &        0.001     &    12.624  &         0.000        &        0.009    &        0.013     \\
\textbf{Temperature}                                        &      -0.0027  &        0.002     &    -1.207  &         0.228        &       -0.007    &        0.002     \\
\textbf{Relative humidity}                                  &      -0.0180  &        0.002     &    -9.309  &         0.000        &       -0.022    &       -0.014     \\
\textbf{Solar radiation}                                    &      -0.0803  &        0.002     &   -33.360  &         0.000        &       -0.085    &       -0.076     \\
\textbf{Wind speed}                                         &      -0.1320  &        0.004     &   -37.219  &         0.000        &       -0.139    &       -0.125     \\
\bottomrule
\end{tabular}
\caption{Robust linear Model Regression Results.}
\label{tab:Robust_table}
\end{center}
\end{table}
\end{landscape}

%
%\begin{figure}[h]
%\centering
%\makebox[\textwidth][c]{\includegraphics[width=0.95\textwidth]{../plots/Residual_plots_MLR_PM2.5}}
%\caption{Residual plots for the PM$_{2.5}$ MLR model.}
%\label{fig:residuals_MLR_PM2.5}
%\end{figure}

% PM10
%\begin{landscape}
%\begin{table}[h]
%\begin{center}
%\begin{tabular}{lclc}
%\toprule
%\textbf{Dep. Variable:}                           & PM$_{10}$, Torkel Knutssonsgatan & \textbf{  R-squared:         } &     0.755   \\
%\textbf{Model:}                                   &                 OLS                 & \textbf{  Adj. R-squared:    } &     0.755   \\
%\textbf{Method:}                                  &            Least Squares            & \textbf{  F-statistic:       } & 2.026e+04   \\
%\textbf{Date:}                                    &           Thu, 11 Aug 2022          & \textbf{  Prob (F-statistic):} &     0.00    \\
%\textbf{Time:}                                    &               23:46:52              & \textbf{  Log-Likelihood:    } &   -10505.   \\
%\textbf{No. Observations:}                        &                 26328               & \textbf{  AIC:               } & 2.102e+04   \\
%\textbf{Df Residuals:}                            &                 26323               & \textbf{  BIC:               } & 2.106e+04   \\
%\textbf{Df Model:}                                &                     4               & \textbf{                     } &             \\
%\textbf{Covariance Type:}                         &              nonrobust              & \textbf{                     } &             \\
%\bottomrule
%\end{tabular}
%\begin{tabular}{lcccccc}
%                                                  & \textbf{coef} & \textbf{std err} & \textbf{t} & \textbf{P$> |$t$|$} & \textbf{[0.025} & \textbf{0.975]}  \\
%\midrule
%\textbf{intercept}                                &       0.0656  &        0.009     &     6.974  &         0.000        &        0.047    &        0.084     \\
%\textbf{PM$_{10}$, Torkel Knutssonsgatan lag1} &       0.6335  &        0.007     &    96.861  &         0.000        &        0.621    &        0.646     \\
%\textbf{PM$_{10}$, Torkel Knutssonsgatan lag2} &       0.1072  &        0.006     &    17.494  &         0.000        &        0.095    &        0.119     \\
%\textbf{PM$_{10}$, Hornsgatan lag1}            &       0.1234  &        0.004     &    30.778  &         0.000        &        0.116    &        0.131     \\
%\textbf{PM$_{10}$, Norr Malma lag1}            &       0.0808  &        0.004     &    19.644  &         0.000        &        0.073    &        0.089     \\
%\bottomrule
%\end{tabular}
%\begin{tabular}{lclc}
%\textbf{Omnibus:}       & 7362.227 & \textbf{  Durbin-Watson:     } &     1.976   \\
%\textbf{Prob(Omnibus):} &   0.000  & \textbf{  Jarque-Bera (JB):  } & 122434.478  \\
%\textbf{Skew:}          &  -0.906  & \textbf{  Prob(JB):          } &      0.00   \\
%\textbf{Kurtosis:}      &  13.408  & \textbf{  Cond. No.          } &      21.8   \\
%\bottomrule
%\end{tabular}
%\caption{OLS Regression Results for PM$_{10}$}
%\end{center}
%\end{table}
%\end{landscape}
%
%% PM2.5
%\begin{landscape}
%\begin{table}[h]
%\begin{center}
%%\resizebox{\textwidth}{!}{%
%\begin{tabular}{lclc}
%\toprule
%\textbf{Dep. Variable:}                            & PM$_{2.5}$, Torkel Knutssonsgatan & \textbf{  R-squared:         } &     0.927   \\
%\textbf{Model:}                                    &                 OLS                  & \textbf{  Adj. R-squared:    } &     0.927   \\
%\textbf{Method:}                                   &            Least Squares             & \textbf{  F-statistic:       } & 8.370e+04   \\
%\textbf{Date:}                                     &           Thu, 11 Aug 2022           & \textbf{  Prob (F-statistic):} &     0.00    \\
%\textbf{Time:}                                     &               23:20:09               & \textbf{  Log-Likelihood:    } &    4594.4   \\
%\textbf{No. Observations:}                         &                 26328                & \textbf{  AIC:               } &    -9179.   \\
%\textbf{Df Residuals:}                             &                 26323                & \textbf{  BIC:               } &    -9138.   \\
%\textbf{Df Model:}                                 &                     4                & \textbf{                     } &             \\
%\textbf{Covariance Type:}                          &              nonrobust               & \textbf{                     } &             \\
%\bottomrule
%\end{tabular}
%%}
%%\resizebox{\textwidth}{!}{%
%\begin{tabular}{lcccccc}
%                                                   & \textbf{coef} & \textbf{std err} & \textbf{t} & \textbf{P$> |$t$|$} & \textbf{[0.025} & \textbf{0.975]}  \\
%\midrule
%\textbf{intercept}                                 &       0.0386  &        0.003     &    11.092  &         0.000        &        0.032    &        0.045     \\
%\textbf{PM$_{2.5}$, Torkel Knutssonsgatan lag1} &       1.1199  &        0.007     &   167.681  &         0.000        &        1.107    &        1.133     \\
%\textbf{PM$_{2.5}$, Torkel Knutssonsgatan lag2} &      -0.2116  &        0.006     &   -35.111  &         0.000        &       -0.223    &       -0.200     \\
%\textbf{PM$_{2.5}$, Hornsgatan, lag1}           &       0.0340  &        0.004     &     8.867  &         0.000        &        0.026    &        0.041     \\
%\textbf{PM$_{2.5}s$, Norr Malma, lag1}           &       0.0218  &        0.001     &    15.576  &         0.000        &        0.019    &        0.025     \\
%\bottomrule
%\end{tabular}
%%}
%\begin{tabular}{lclc}
%\textbf{Omnibus:}       & 5704.682 & \textbf{  Durbin-Watson:     } &     1.987   \\
%\textbf{Prob(Omnibus):} &   0.000  & \textbf{  Jarque-Bera (JB):  } & 161104.434  \\
%\textbf{Skew:}          &  -0.393  & \textbf{  Prob(JB):          } &      0.00   \\
%\textbf{Kurtosis:}      &  15.093  & \textbf{  Cond. No.          } &      21.3   \\
%\bottomrule
%\end{tabular}
%\caption{OLS Regression Results for PM$_{2.5}$}
%\end{center}
%\end{table}
%\end{landscape}

\section{Histogram and qq-plots of the forecast errors, Box-Pierce test results, and results from the Dunn's test}
\label{chapt:appendix_C}

The results from the Box-Pierce tests for all models are summarized in \cref{tab:boxpierce} below. 
\begin{table}[h]
\small
\centering
\caption{Results from the Box-Pierce tests.}
\label{tab:boxpierce}
\begin{tabular}{@{}lll@{}}
\toprule
Model                  & $Q_{\text{BP}}$ & $p$-value             \\ \midrule
Robust MLR model       & 1520.07         & $\ll0.001$ \\
Dense model            & 866.76          & $\ll0.001$  \\
Dense model (6h)       & 607.94          & $\ll0.001$ \\
Simple RNN model (12h) & 238.62          & $\ll0.001$  \\
LSTM model (12h)       & 315.34          & $\ll0.001$  \\
GRU model (24h)        & 506.06          & $\ll0.001$ \\ \bottomrule
\end{tabular}
\end{table}

\noindent
The $p$-values from the Dunn's post hoc test with Bonferroni adjustment for the MSE and the ME measures are given in \cref{tab:Dunn_MSE} and \cref{tab:Dunn_ME}, respectively. 

%MSE
% Please add the following required packages to your document preamble:
% \usepackage{booktabs}
% \usepackage{graphicx}
\begin{table}[h]
\centering
\caption{$p$-values for the pairwise comparisons from the Dunn's post hoc test for the MSE. }
\label{tab:Dunn_MSE}
\resizebox{\textwidth}{!}{%
\begin{tabular}{@{}lllllll@{}}
\toprule
Model            & Robust MLR & Dense      & Dense (6h) & Simple RNN (12h) & LSTM (12h) & GRU (24h)  \\ \midrule
Robust MLR       & 1.0        & 0.00752    & $\ll0.001$ & $\ll0.001$       & 0.00209    & $\ll0.001$ \\
Dense            & 0.00752    & 1.0        & $\ll0.001$ & $\ll0.001$       & 1.0        & $\ll0.001$ \\
Dense (6h)       & $\ll0.001$ & $\ll0.001$ & 1.0        & 1.0              & $\ll0.001$ & 1.0        \\
Simple RNN (12h) & $\ll0.001$ & $\ll0.001$ & 1.0        & 1.0              & $\ll0.001$ & 1.0        \\
LSTM (12h)       & 0.00209    & 1.0        & $\ll0.001$ & $\ll0.001$       & 1.0        & $\ll0.001$ \\
GRU (24h)        & $\ll0.001$ & $\ll0.001$ & 1.0        & 1.0              & $\ll0.001$ & 1.0        \\ \bottomrule
\end{tabular}%
}
\end{table}

%ME
\begin{table}[h]
\centering
\caption{$p$-values for the pairwise comparisons from the Dunn's post hoc test for the ME. }
\label{tab:Dunn_ME}
\resizebox{\textwidth}{!}{%
\begin{tabular}{lllllll}
\hline
Model            & Robust MLR & Dense      & Dense (6h) & Simple RNN (12h) & LSTM (12h) & GRU (24h)  \\ \hline
Robust MLR       & 1.0        & $\ll0.001$ & $\ll0.001$ & $\ll0.001$       & $\ll0.001$ & $\ll0.001$ \\
Dense            & $\ll0.001$ & 1.0        & 0.13083    & $\ll0.001$       & $\ll0.001$ & $\ll0.001$ \\
Dense (6h)       & $\ll0.001$ & 0.13083    & 1.0        & $\ll0.001$       & $\ll0.001$ & $\ll0.001$ \\
Simple RNN (12h) & $\ll0.001$ & $\ll0.001$ & $\ll0.001$ & 1.0              & 1.0        & 0.57372    \\
LSTM (12h)       & $\ll0.001$ & $\ll0.001$ & $\ll0.001$ & 1.0              & 1.0        & 0.83808    \\
GRU (24h)        & $\ll0.001$ & $\ll0.001$ & $\ll0.001$ & $\ll0.001$       & 0.83808    & 1.0        \\ \hline
\end{tabular}%
}
\end{table}

Quantile-quantile plots and histogram plots of the forecast errors are shown, respectively, in \cref{fig:qq-plot-errors} and \cref{fig:histogram_errors} below. It can be inferred from these plots that none of the models generated forecasts with the desired structure of the errors (i.e., Gaussian white noise).

% forecast errors normal probability plots
\begin{figure}[h]
\centering
\makebox[\textwidth][c]{\includegraphics[width=1.025\textwidth]{../plots/qq_pred_errors_all}}
\caption{Normal probability plots for the forecast errors.}
\label{fig:qq-plot-errors}
%\end{figure}
%\begin{figure}[]
%\bigskip
%\centering
\vspace*{\floatsep}% https://tex.stackexchange.com/q/26521/5764
\makebox[\textwidth][c]{\includegraphics[width=1.025\textwidth]{../plots/pred_errors_hist_all}}
\caption{Histogram for the forecast errors.}
\label{fig:histogram_errors}
\end{figure}

\end{document}