\thispagestyle{empty}
%In today's world, ambient air pollution is a pressing environmental concern, 
%and the majority of the world's population is regularly exposed to air pollution levels higher than what is deemed healthy.
%and to mitigate the negative effects, air quality is regularly monitored. Such monitoring can produce large amounts of data, and this in turn enables the development of statistical and machine learning techniques for air quality forecasts. However, due to the complex nature of air pollution, such data can be challenging to utilize, but machine learning techniques, especially deep neural networks, have shown promising results. The research problem in this work is therefore; \textit{To capture and model the complex dynamics of air pollution with machine learning methods, with an emphasis on deep neural networks}. The research question of the thesis is; \textit{How can machine learning, in particular deep neural networks, be used to forecast air pollution levels and pollution peaks?} In the research question, an emphasis it put on pollution peaks, as these are the episodes when existing forecasting models tend to give the largest prediction errors. Historical data from air monitoring sensors were utilized to train several neural network architectures, as well as a more straightforward multiple linear regression model, for forecasting background levels of nitrogen dioxide one hour ahead in the central of Stockholm. Several evaluation metrics showed that the neural network models outperformed the multiple linear regression model, however, none of the models had the desired structure of the forecast errors, and all models failed to successfully capture sudden pollution peaks. Nevertheless, the results point to an advantage for the more complex neural network models, and further advances in the field of machine learning, together with higher resolution data, have the potential to improve air quality forecasts even more and cross conventional forecasting limits.

In today's world, where air pollution has become a ubiquitous problem, city air is normally monitored. Such monitoring can produce large amounts of data, and this enables the development of statistical and machine learning techniques for modeling and forecasting air quality. However, the complex nature of air pollution makes such data a challenge to fully utilize. To this end, machine learning methods, especially deep neural networks, have in recent years emerged as a promising technology for more accurate predictions of air pollution levels, and the research problem in this work is; \textit{To capture and model the complex dynamics of air pollution with machine learning methods, with an emphasis on deep neural networks}. Connected to the research problem is the research question; \textit{How can machine learning, in particular deep neural networks, be used to forecast air pollution levels and pollution peaks?} An emphasis is put on pollution peaks, as these are the episodes when existing forecasting models tend to give the largest prediction errors. In this work, historical data from air monitoring sensors were utilized to train several neural network architectures, as well as a more straightforward multiple linear regression model, for forecasting background levels of nitrogen dioxide in the center of Stockholm. Several evaluation metrics showed that the neural network models outperformed the multiple linear regression model, however, none of the models had the desired structure of the forecast errors, and all models failed to successfully capture sudden pollution peaks. Nevertheless, the results point to an advantage for the more complex neural network models, and further advances in the field of machine learning, together with higher resolution data, have the potential to improve air quality forecasts even more and cross conventional forecasting limits.
\\ 

\noindent
\emph{\textbf{Keywords}:} Air pollution forecasts, neural networks, linear regression, time series analysis, nitrogen dioxide\\


%Air pollution monitoring typically generates large amounts of data, which enables the development of statistical and machine learning techniques for air quality forecasts. However, the complex nature of air pollution makes such data a challenge to fully utilize. However, machine learning techniques, especially deep neural networks, have shown promising results in this area, and 

