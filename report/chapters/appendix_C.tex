\label{chapt:appendix_C}

The results from the Box-Pierce tests for all models are summarized in \cref{tab:boxpierce} below. 
\begin{table}[h]
\small
\centering
\caption{Results from the Box-Pierce tests.}
\label{tab:boxpierce}
\begin{tabular}{@{}lll@{}}
\toprule
Model                  & $Q_{\text{BP}}$ & $p$-value             \\ \midrule
Robust MLR model       & 1520.07         & $\ll0.001$ \\
Dense model            & 866.76          & $\ll0.001$  \\
Dense model (6h)       & 607.94          & $\ll0.001$ \\
Simple RNN model (12h) & 238.62          & $\ll0.001$  \\
LSTM model (12h)       & 315.34          & $\ll0.001$  \\
GRU model (24h)        & 506.06          & $\ll0.001$ \\ \bottomrule
\end{tabular}
\end{table}

\noindent
The $p$-values from the Bonferroni-Dunn post-hoc tests for the MSE are given in \cref{tab:Dunn_MSE}. 

%MSE
% Please add the following required packages to your document preamble:
% \usepackage{booktabs}
% \usepackage{graphicx}
\begin{table}[h]
\centering
\caption{$p$-values for the pairwise comparisons from the Bonferroni-Dunn post-hoc test for the MSE. }
\label{tab:Dunn_MSE}
\resizebox{\textwidth}{!}{%
\begin{tabular}{@{}lllllll@{}}
\toprule
Model            & Robust MLR & Dense      & Dense (6h) & Simple RNN (12h) & LSTM (12h) & GRU (24h)  \\ \midrule
Robust MLR       & 1.0        & 0.00665    & $\ll0.001$ & $\ll0.001$       & 0.00170    & $\ll0.001$ \\
Dense            & 0.00665    & 1.0        & $\ll0.001$ & $\ll0.001$       & 1.0        & $\ll0.001$ \\
Dense (6h)       & $\ll0.001$ & $\ll0.001$ & 1.0        & 1.0              & $\ll0.001$ & 0.04855        \\
Simple RNN (12h) & $\ll0.001$ & $\ll0.001$ & 1.0        & 1.0              & $\ll0.001$ & 0.00044        \\
LSTM (12h)       & 0.00170    & 1.0        & $\ll0.001$ & $\ll0.001$       & 1.0        & 0.00001 \\
GRU (24h)        & $\ll0.001$ & $\ll0.001$ & 0.04855        & 0.00044              & 0.00001 & 1.0        \\ \bottomrule
\end{tabular}%
}
\end{table}

%%ME
%\begin{table}[h]
%\centering
%\caption{$p$-values for the pairwise comparisons from the Dunn's post hoc test for the ME. }
%\label{tab:Dunn_ME}
%\resizebox{\textwidth}{!}{%
%\begin{tabular}{lllllll}
%\hline
%Model            & Robust MLR & Dense      & Dense (6h) & Simple RNN (12h) & LSTM (12h) & GRU (24h)  \\ \hline
%Robust MLR       & 1.0        & $\ll0.001$ & $\ll0.001$ & $\ll0.001$       & $\ll0.001$ & $\ll0.001$ \\
%Dense            & $\ll0.001$ & 1.0        & 0.13083    & $\ll0.001$       & $\ll0.001$ & $\ll0.001$ \\
%Dense (6h)       & $\ll0.001$ & 0.13083    & 1.0        & $\ll0.001$       & $\ll0.001$ & $\ll0.001$ \\
%Simple RNN (12h) & $\ll0.001$ & $\ll0.001$ & $\ll0.001$ & 1.0              & 1.0        & 0.57372    \\
%LSTM (12h)       & $\ll0.001$ & $\ll0.001$ & $\ll0.001$ & 1.0              & 1.0        & 0.83808    \\
%GRU (24h)        & $\ll0.001$ & $\ll0.001$ & $\ll0.001$ & $\ll0.001$       & 0.83808    & 1.0        \\ \hline
%\end{tabular}%
%}
%\end{table}

Quantile-quantile plots and histogram plots of the forecast errors are shown, respectively, in \cref{fig:qq-plot-errors} and \cref{fig:histogram_errors} below. It can be inferred from these plots that none of the models generated forecasts with the desired structure of the errors (i.e., Gaussian white noise).

% forecast errors normal probability plots
\begin{figure}[h]
\centering
\makebox[\textwidth][c]{\includegraphics[width=1.025\textwidth]{../plots/qq_pred_errors_all}}
\caption{Normal probability plots for the forecast errors.}
\label{fig:qq-plot-errors}
%\end{figure}
%\begin{figure}[]
%\bigskip
%\centering
\vspace*{\floatsep}% https://tex.stackexchange.com/q/26521/5764
\makebox[\textwidth][c]{\includegraphics[width=1.025\textwidth]{../plots/pred_errors_hist_all}}
\caption{Histogram for the forecast errors.}
\label{fig:histogram_errors}
\end{figure}
