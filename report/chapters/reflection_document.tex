In this work, several regression models have been developed to forecast background levels of air pollution in the center of Stockholm. The necessary theoretical basis has been obtained from highly regarded sources, both textbooks and peer reviewed scientific articles, which have been reviewed and also used as inspiration for the work conducted. The scientific literature for time series analysis and forecasting/prediction is vast, and the studies as well as textbook sections considered had to be limited due to time constraints. Nevertheless, necessary theory for implementing and evaluating the models were utilized. 

The contribution this thesis makes stems from the fact that a regression modeling approach to air pollution forecasts is lacking in Sweden. To this end, the prediction models developed have contributed to the body of knowledge within air quality forecasts. Model evaluation have also shown weaknesses in the models and other potential difficulties; this can serve as valuable pointers to further developments within the area. The ethical considerations for the thesis are practically non-existent, but societal implications have been pointed out and elaborated on.

The planning of the study began with collecting relevant scientific literature and textbooks. Following this, the data needed were collected and examined. This was a crucial step in the process, as data quantity (and quality) is important, particularly for deep neural network models. Being such an important step, the data collection and examining part took a considerable amount of time, and some steps could have been automated, thereby saving time. The deep neural network models also take a lot of time to train and tune, and faster GPU's would have helped to further save time. Nevertheless, the project was subsequently finished, though the final presentation and thesis version became delayed due to a COVID-19 infection. 

Having a background in chemistry, the topic of the thesis felt natural, and I am passionate about applying statistical and machine learning models to problems in the natural sciences. This thesis topic has let me do exactly that, and upon completing the thesis, only further interest has been sparked. What I have learned will be very valuable for future work, as it has tied together many aspects of a data science project (data gathering, model choice and fitting, model evaluation, and so on). Also it has given me training in writing scientifically, as well as time management and generally given me insights how to structure a data science problem to reach set up goals within the time frames given. Overall I am satisfied with the results, even though the models were not able to accurately predict pollution peaks, and in light of the research question the model performance was (somewhat) unsatisfactory. Nevertheless, the implemented models showed a promising potential, with relatively good predictive capability at shorter forecast horizons, and this can still be regarded as a successful implementation of an air pollution forecasting system.  