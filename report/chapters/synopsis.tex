\thispagestyle{empty}

\paragraph{Background}
To mitigate the harmful effects of air pollution, air quality is regularly monitored. Such monitoring produces large amounts of data, and this enables the development of statistical and machine learning techniques for modeling and forecasting air pollution levels.
\paragraph{Problem}
Air pollution is a complex phenomenon depending on many factors, and the data from air monitoring has both temporal and spatial dependencies. This makes modeling and forecasting a challenge, and the research problem in this work is: \textit{To capture and model the complex dynamics of air pollution with machine learning methods, with an emphasis on deep neural networks}.

\paragraph{Research Question}
The research question for the thesis is: \textit{How can machine learning, in particular deep neural networks, be used to forecast air pollution levels and pollution peaks?} The research question emphasizes pollution peaks, as these are much harder to accurately predict and forecast than when pollution levels are lower.  %Also, sudden jerky leaps, where pollution levels quickly rise tend to give strongly underestimated predictions, and this can have undesired consequences
%
%All type of forecasts are always imprecise, and generally for air pollution forecasts, the largest errors are seen during episodes with high pollution levels. Also, sudden jerky leaps, where pollution levels quickly rise tend to give strongly underestimated predictions, and this can have undesired consequences. The research question in this work are therefore as follows: 

\paragraph{Method}
Data was downloaded from SMHI's centralized database for air measurements, carefully examined, and thereafter preprocessed before a linear regression and several deep neural network models were fit. All models were trained with historical data and later evaluated on the most recent (test) data. An important aspect of model evaluation was a close examination of the forecast errors. The Python programming language was used together with libraries for scientific programming and data science/machine learning.

\paragraph{Result}
All deep learning models outperformed the linear regression model. However, for all models, the structure of the forecast errors was not as desired, and this warrants further model refinements. The structure of the errors was due to the inability of the models to capture pollution peaks well. Though the deep neural network models showed promising potential, in light of the research question, the results were unsatisfactory. 

\paragraph{Discussion}
The forecast horizons in this work (one hour) are very short-term, which limits usability. To this end, however, adaptations to extend the forecast horizons are possible. Further adaptations to include more than just one air pollutant is also an option; this would result in comprehensive forecasts that could be of great value to public health authorities and policymakers, as it could permit early interventions to protect public health and vulnerable groups.
